% Title page.
\title[Aula 08]{Massas d'água}
\subtitle{(AT, ACAS, APAN, AIA, AAF)}
\author[Filipe Fernandes]{Filipe P. A. Fernandes}
\institute[unimonte]{Centro Universitário Monte Serrat}
\date[Outubro 2013]{25 de Outubro 2013}

\logo{\includegraphics[scale=0.15]{../common/university_logo.png}}

\begin{document}

% The title page frame.
\begin{frame}[plain]
  \titlepage
\end{frame}

\section*{Outline}
\begin{frame}
\tableofcontents
\end{frame}

\section{Massas d'água}
\begin{frame}
\frametitle{Massas d'água}
  \begin{itemize}[<+-| alert@+>]
    \item Definição: Corpos de água com características particulares de
          temperatura e salinidade.
    \item Princípio: A característica conservativa da temperatura (T) e
          salinidade (S).
    \item Aplicações:
      \begin{enumerate}[<+-| alert@+>]
        \item[a)] Identificar as massas d'água e suas fontes.
        \item[b)] Estudar Mistura de águas.
      \end{enumerate}
  \end{itemize}
\end{frame}


\begin{frame}
\frametitle{Massas d'água}
  \begin{itemize}[<+-| alert@+>]
    \item A maioria das trocas de calor e água doce entre o oceano e a
          atmosfera ocorrem na camada superficial dos oceanos (até 150 m);
    \item Quando uma parcela de água é removida da camada superficial, sua
          temperatura e salinidade permanecem inalterados até que a mesma
          retorne a superfície;
    \item No entanto, estes movimentos são lentos e a medição direta destes
          deslocamentos é difícil.
  \end{itemize}
\end{frame}


\begin{frame}
\frametitle{Massas d'água}
{\small
\begin{block}{}
  É um corpo de água com uma história de formação comum e que tem sua origem em
  uma região específica do oceano. Massas de água são entidades físicas que
  possuem um volume mensurável, ocupando uma parcela finita no oceano. Na sua
  região de formação, elas ocupam uma parcela exclusiva de uma determinada
  parte do oceano. Em outras regiões, elas dividem o oceano com outras massas
  de água, com as quais se misturam. O volume total de uma massa de água é dado
  pela soma de todos os seus elementos, independente da sua localização
\end{block}
}
\end{frame}


\begin{frame}
\frametitle{Conceitos básicos}
  \begin{itemize}[<+-| alert@+>]
    \item Um único ponto é chamado de ``água tipo''.
    \item Uma curva suave e chamada de massa d'água.
  \end{itemize}
  \pause
  \begin{center}
    \includegraphics[scale=0.55]{../figures/massa_dagua_mistura_01.png}
  \end{center}
\end{frame}


\begin{frame}
\frametitle{Mistura de 3 massas d'água}
  \begin{center}
    \includegraphics[scale=0.6]{../figures/massa_dagua_mistura_02.png}
  \end{center}
\end{frame}


\begin{frame}
\frametitle{Formação da curva TS}
  \begin{center}
    \includegraphics[scale=0.45]{../figures/massa_dagua_mistura_03.png}
  \end{center}
\end{frame}


\begin{frame}
\frametitle{Água com um ``range'' de TS}
{\small
\begin{block}{}
  Em geral a mistura de duas massas d'água mostra uma linha reta num diagrama TS.  Mas isso não é sempre verdade!
\end{block}
}
  \begin{center}
    \includegraphics[scale=0.6]{../figures/acas.png}
  \end{center}
\end{frame}


\begin{frame}
\frametitle{\small Podemos usar temperatura da água {\it in situ} e salinidade para
desenhar um diagrama TS e comparar massas d'agua?}
  \begin{center}
    \includegraphics[scale=0.5]{../figures/TS_question.png}
  \end{center}
\end{frame}


\begin{frame}
\frametitle{Massas d'água superiores}
  \begin{center}
    \includegraphics[scale=0.45]{../figures/upper_water_masses.png}
  \end{center}
\end{frame}


\begin{frame}
\frametitle{Formação de água central}
  \begin{center}
    \includegraphics[scale=0.6]{../figures/agua_central.png}
  \end{center}
\end{frame}


\begin{frame}
\frametitle{Tipos de mistura}
  \begin{center}
    \includegraphics[scale=0.45]{../figures/mixing_types.png}
  \end{center}
\end{frame}


\begin{frame}
\frametitle{\small Qual tipo de mistura forma a ACAS?}
  \begin{center}
    \includegraphics[scale=0.45]{../figures/TS_medio.png}
  \end{center}
\end{frame}


\begin{frame}
\frametitle{Formação de água intermediária}
{\small
\begin{block}{}
  Formada em zonas subpolares onde a precipitação excede a evaporação pelo
  afundamento da massa de água relativamente mais densa na superfície
  de zonas de convergência subpolares.
\end{block}
}
  \begin{center}
    \includegraphics[scale=0.25]{../figures/meridional_watermasses.png}
  \end{center}
\end{frame}


\begin{frame}
\frametitle{Massas d'água intermediárias}
  \begin{center}
    \includegraphics[scale=0.6]{../figures/massa_intermediaria.png}
  \end{center}
\end{frame}


\begin{frame}
\frametitle{Formação de águas profundas e de fundo}
{\small
\begin{block}{}
  Formada em zonas polares por água de baixa temperatura e alta salinidade.
\end{block}
}
  \begin{center}
    \includegraphics[scale=0.5]{../figures/nadw.png}
  \end{center}
\end{frame}


\begin{frame}
\frametitle{Massas d'água profundas}
  \begin{center}
    \includegraphics[scale=0.45]{../figures/nadw_02.png}
  \end{center}
\end{frame}


\begin{frame}
\frametitle{Massas d'água profundas}
  \begin{center}
    \includegraphics[scale=0.75]{../figures/nadw_03.png}
  \end{center}
\end{frame}


\begin{frame}
\frametitle{Massas d'água de fundo}
  \begin{center}
    \includegraphics[scale=0.6]{../figures/aaf.png}
  \end{center}
\end{frame}


\begin{frame}
\frametitle{Polynya}
  \begin{center}
    \includegraphics[scale=0.7]{../figures/polynya.png}
  \end{center}
\end{frame}


\begin{frame}
\frametitle{Mar de Weddell -- 1973--1976}
  \begin{center}
    \includegraphics[scale=0.55]{../figures/polynya_02.png}
  \end{center}
\end{frame}


\begin{frame}
\frametitle{Massas d'água fundo}
  \begin{center}
    \includegraphics[scale=0.45]{../figures/massas_fundo.png}
  \end{center}
\end{frame}


\begin{frame}
\frametitle{Principais massas d'água nos 3 oceanos}
  \begin{center}
    \includegraphics[scale=0.6]{../figures/massas_3oceanos.png}
  \end{center}
\end{frame}


\begin{frame}
\frametitle{Diagrama TS para o Atlântico Sul}
  \begin{center}
    \includegraphics[scale=0.45]{../figures/TS_atlantico_sul.png}
  \end{center}
\end{frame}


\end{document}
