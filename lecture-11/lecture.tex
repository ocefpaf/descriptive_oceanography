% Title page.
\title[Aula 10]{Correntes oceânicas}
% \subtitle{}
\author[Filipe Fernandes]{Filipe P. A. Fernandes}
\institute[unimonte]{Centro Universitário Monte Serrat}
\date[Novembro 2012]{26 de Novembro 2012}

\logo{\includegraphics[scale=0.15]{../common/university_logo.png}}

\begin{document}

% The title page frame.
\begin{frame}[plain]
  \titlepage
\end{frame}

\section*{Outline}
\begin{frame}
\tableofcontents
\end{frame}

\section{Correntes Oceânicas}
\begin{frame}
\frametitle{Correntes Oceânicas}
\small{
  \begin{itemize}[<+-| alert@+>]
    \item A circulação oceânica é o resultado de um balanço de forças
          (pelo menos 2).
    \item A dinâmica de fluidos mostra que de acordo com a região de
          estudo, ocorre um determinado tipo de balanço.
    \item As forças de atrito só são importantes nos limites dos
          oceanos (atmosfera e contornos).
    \item Na região mais interior do oceano e abaixo da camada
          superficial, a força de atrito, quando comparada com a força
          do gradiente de pressão (devido à variações horizontais no
          campo de densidade), pode ser desprezada.
  \end{itemize}
  }
\end{frame}

\subsection{O efeito de Coriolis}
\begin{frame}
 \frametitle{Coriolis}
 \begin{block}{}
  HN: Deflete para a direita.

  HS: Deflete para a esquerda.
 \end{block}
  \href{http://www.physics.orst.edu/~mcintyre/coriolis/East_GIF.html}{Lest}

  \href{http://www.physics.orst.edu/~mcintyre/coriolis/North_Lon_GIF.html}{Norte}

  \href{http://www.physics.orst.edu/~mcintyre/coriolis/North_Pole_GIF.html}{Pólo}

\end{frame}

\begin{frame}
 \frametitle{Força de Coriolis}
 \small{
 \begin{itemize}[<+-| alert@+>]
  \item É proporcional a magnitude da velocidade do
        fluxo.
  \item É direcionada perpendicularmente ao fluxo.
  \item Age defletindo o fluxo para a esquerda no HS e para a direita
        no HN.
  \item É máxima nos polos e nulo na região equatorial.
  \item \( f = 2\Omega\sin(\theta) \), \( \Omega = \frac{2\pi}{T_S} \)
        onde:
  \item $\Omega = 7,29 \times 10^{-5}$ s${-1}$ é a velocidade angular da Terra.
  \item $\theta$ é a latitude.
  \item $T_S = 86164,09$ s (23 h 56 m 4.1 s) é o dia sideral.
 \end{itemize}
 }
\end{frame}


\begin{frame}
 \frametitle{Coriolis}
 \begin{block}{}
    Se a Terra fosse cilíndrica haveria força de Coriolis?
 \end{block}
  \pause
 \begin{block}{}
  (a) Um míssil é disparado do equador em direção ao sul.  Explique o
  que aconteceria com a sua trajetória em relação a Terra.

  (b) Em que direção uma corrente seria defletida pela força de
  Coriolis se ela fluísse inicialmente para (i) leste a
  45\textdegree{N} e (ii) oeste no equador?
 \end{block}
  \href{http://www.youtube.com/watch?v=1PWjOMrPkrI}{Vídeo}
\end{frame}

\begin{frame}
\frametitle{Correntes Oceânicas}
  \begin{block}{}
    A força do gradiente de pressão, no entanto, não é a única força
    que ocorre nesta região no interior dos oceanos.
  \end{block}
  \pause
  \begin{block}{}
    Como os movimentos nos oceanos são lentos, a rotação da Terra passa
    a ser relevante e influencia os movimentos através de uma força
    aparente, denominada de força de Coriolis.
  \end{block}
\end{frame}

\subsection{Revisão de Ekman}
\begin{frame}
\frametitle{Revisão de Ekman}
\small{
 \begin{block}{}
  As correntes geradas pelo vento não afetam apenas a camada de Ekman,
  uma vez que os transportes na camada de Ekman também atuam como um
  importante mecanismo gerador de fluxo para as camadas profundas,
  através do bombeamento de Ekman.
 \end{block}
\pause
\begin{block}{}
  Existem algumas regiões do globo onde o transporte na camada de Ekman é convergente e outras onde ele é divergente. Isto induz movimentos verticais abaixo da camada de Ekman, que atuam para repor ou remover as massas de água associadas.
\end{block}
}
\end{frame}

\begin{frame}
\frametitle{Revisão de Ekman}
 \begin{block}{}
 \[ w = \frac{1}{f\rho} \left( \pd{\tau^y}{x} - \pd{\tau^x}{y} \right)\]
  \vspace{1cm}
 $w \uparrow$ sucção de Ekman (ressurgência).

 $w \downarrow$ bombeamento de Ekman (subsidência).
 \end{block}
\end{frame}


\begin{frame}
  \frametitle{Revisão de Ekman}
  \begin{center}
    \includegraphics[scale=0.34]{../figures/ekman_bombeamento.png}
  \end{center}
\end{frame}


\subsection{Balanço Geostrófico}
\begin{frame}
  \frametitle{Balanço Geostrófico}
  \begin{itemize}[<+-| alert@+>]
    \item Ocorre amplamente no interior do oceano e da atmosfera;
    \item É representado pelo equilíbrio entre a Força do Gradiente de Pressão
          e a Força de Coriolis;
    \item É um equilíbrio estacionário;
    \item A pressão em um ponto no interior do oceano é função do peso de água
          acima deste, que é função da densidade $\rho(S, T, P)$ e da altura da
          coluna d'água.
  \end{itemize}
\end{frame}

\begin{frame}
  \frametitle{Balanço Geostrófico}
  \small{
  \begin{block}{Conservação do momento completa}
    \[
       \pd{\mathbf{u}}{t} + (\mathbf{u}.\nabla_h)\mathbf{u} +
       w\pd{\mathbf{u}}{z} + \mathbf{k} \times f\mathbf{u} = -
       \frac{1}{\rho_o}\nabla_hp + \pd{}{z}\left(K_M\pd{\mathbf{u}}{z}\right) +
       \mathbf{F}
    \]
  \end{block}
  }
\end{frame}

\begin{frame}
  \frametitle{Balanço Geostrófico}
  \begin{block}{Força de Coriolis -- Força Gradiente de Pressão}
    \hspace{1cm} \( fu = -\frac{1}{\rho}\pd{p}{y} \), \hspace{0.5cm} \( fv = -\frac{1}{\rho}\pd{p}{x} \)
  \end{block}
\end{frame}


\begin{frame}
 \frametitle{Altura dinâmica}
 \begin{block}{}
   Uma quantidade muito usada em oceanografia é a altura dinâmica
   (dimensão m$^2$s$^{-2}$), que a altura estérica escalonada pela
   aceleração da gravidade.  A altura estérica (dimensão de
   comprimento), entretanto,  fornece uma medida comparável com
   comprimento e a sua distribuição espacial pode ser interpretada como
   o formato da superfície dos oceanos.
 \end{block}
\end{frame}

\begin{frame}
 \frametitle{Altura dinâmica}
 \begin{block}{}
  \(D(p_1, p_2) = \int^{p_1}_{p_1}\delta(S, T, P)dp\)
  \hspace{1cm} \( \left[\frac{\text{m}^3}{\text{kg}} \frac{\text{kg m s}^{-2}}{\text{m}^2}\right]\)
 \end{block}
\end{frame}


\begin{frame}
  \frametitle{Altura dinâmica}
  \begin{center}
    \includegraphics[scale=0.24]{../figures/dynamic_height.png}
  \end{center}
\end{frame}


\subsection{Circulação Termohalina}
\begin{frame}
  \frametitle{Circulação Termohalina}
 \begin{itemize}[<+-| alert@+>]
  \item Distribuição de T/S nos oceanos.
  \item Circulação termohalina geral.
  \item Correntes de contorno oeste profundas.
 \end{itemize}
\end{frame}


\begin{frame}
  \frametitle{Conveyor belt}
  \begin{center}
    \includegraphics[scale=0.40]{../figures/thermohaline_circulation.png}
  \end{center}
\end{frame}


\begin{frame}
  \frametitle{Formação de água de fundo}
  \begin{itemize}[<+-| alert@+>]
    \item T/S das águas superficiais são determinadas por
          resfriamento/aquecimento e evaporação/precipitação.
    \item Uma vez isolados da superfície, T/S são conservados.
  \end{itemize}
\end{frame}


\begin{frame}
  \frametitle{Formação de água de fundo}
  \begin{center}
    \includegraphics[scale=0.45]{../figures/deep_water_formation.png}
  \end{center}
\end{frame}


\subsection{Sumário}
\begin{frame}
\frametitle{Sumário}
  \begin{itemize}[<+-| alert@+>]
    \item A circulação gerada pelo vento (principalmente associada aos giros
          subtropicais), apresenta uma forte simetria entre o HN e HS.
    \item A circulação envolvendo os 1000 m do oceano é mais rápida, ocorre em
          escalas de bacias e medidas diretas podem ser usadas para a sua
          avaliação.
    \item A circulação termohalina é lenta, ocorre em escala global e é
          geralmente estudada através do conceito de massas de água.
  \end{itemize}
\end{frame}


\begin{frame}
  \frametitle{Correntes}
  \begin{center}
    \includegraphics[scale=1.2]{../figures/currents_cartoon.png}
  \end{center}
  \href{http://www.youtube.com/watch?v=BtCwjfU2Rro}{Vídeo}
\end{frame}


\begin{frame}
  \frametitle{Dever de casa}
  \pause
    Dúvidas?
\end{frame}

\end{document}
