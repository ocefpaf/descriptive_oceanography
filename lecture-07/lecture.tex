% Title page.
\title[Aula 07]{4. Distribuições e variações temporais das propriedades físicas}
\subtitle{(horizontais e verticais)}
\author[Filipe Fernandes]{Filipe P. A. Fernandes}
\institute[unimonte]{Centro Universitário Monte Serrat}
\date[Outubro 2013]{25 de Outubro 2013}

\logo{\includegraphics[scale=0.15]{../common/university_logo.png}}

\begin{document}

% The title page frame.
\begin{frame}[plain]
  \titlepage
\end{frame}

\section*{Outline}
\begin{frame}
\tableofcontents
\end{frame}


\section{Revisão}
\begin{frame}
\frametitle{Revisão}
  \begin{itemize}[<+-| alert@+>]
    \item[1] Introdução à Oceanografia Física
    \item[1.1] Objetivos da Oceanografia Física
    \item[1.2] Divisões da Oceanografia Física
    \item[1.3] Histórico da Oceanografia Física
    \item[2] Dimensões e formas dos oceanos
    \item[3] Propriedades físicas da água do mar
    \item[3.1] Temperatura
    \item[3.2] Salinidade e condutividade
    \item[3.3] Pressão
    \item[3.4] Densidade
    \item[3.5] Efeitos da temperatura, da salinidade e da pressão na densidade
  \end{itemize}
\end{frame}


\begin{frame}
\frametitle{Revisão}
  \begin{itemize}[<+-| alert@+>]
    \item[3.6] Volume específico
    \item[3.7] Propriedades conservativas e não conservativas
    \item[3.8] Equação de estado da água do mar.
    \item[3.9] Gases dissolvidos
    \item[4] Distribuição (horizontal e vertical) e variações temporais das propriedades físicas
    \item[4.1] Distribuição da temperatura
    \item[4.2] Variações temporais da temperatura
    \item[4.3] Distribuição da salinidade
    \item[4.4] Variações temporais da salinidade
  \end{itemize}
\end{frame}


\begin{frame}
\frametitle{Revisão}
  \begin{itemize}[<+-| alert@+>]
    \item[4.5] Distribuição da densidade
    \item[4.6] Distribuição vertical da densidade e picnoclina
    \item[4.7] Estabilidade estática
    \item[4.8] Variações da densidade
  \end{itemize}
\end{frame}


\begin{frame}
\frametitle{Distribuição da densidade}
\footnotesize{
  \begin{itemize}[<+-| alert@+>]
    \item A densidade normalmente aumenta com a profundidade.
    \item Este aumento no entanto não é uniforme (quando removemos o efeito da
          pressão!).
    \item Nas regiões equatoriais e tropicais há normalmente uma camada
          superior fina de densidade praticamente uniforme;
    \item Depois existe uma camada onde a densidade aumenta rapidamente com a
          profundidade -- a Picnoclina;
    \item Abaixo existe a camada profunda onde a densidade aumenta mais
          lentamente com a profundidade.
  \end{itemize}
  }
\end{frame}


\begin{frame}
\frametitle{Distribuição vertical da densidade e picnoclina}
  \begin{center}
    \includegraphics[scale=0.35]{../figures/typical_sigma_t.png}
  \end{center}
\end{frame}


\begin{frame}
\frametitle{Estabilidade Estática}
  \begin{itemize}[<+-| alert@+>]
    \item A taxa de variação da densidade com a profundidade determina a
          Estabilidade Estática da água.
    \item A magnitude da tendência da porção de água a regressar à sua posição
          original é a medida quantitativa da estabilidade (E) da coluna de
          água.
    \item $E = -\frac{1}{\rho}\pd{\rho}{z}$
  \end{itemize}
\end{frame}


\begin{frame}
\frametitle{$E = -\frac{1}{\rho}\pd{\rho}{z}$}
  \begin{itemize}[<+-| alert@+>]
    \item $E > 0$, a coluna d'água é estável;
    \item $E = 0$, a coluna d'água é neutra;
    \item $E < 0$, a coluna d'água é instável;
    \item $E = -\frac{1}{\rho}\pd{\rho}{z} - (\frac{g}{C^2})$;
    \item $N^2 = gE \equiv -g\frac{1}{\rho}\pd{\rho}{z}$;
    \item Notar que quanto maior a estabilidade maior a frequência.
  \end{itemize}
\end{frame}


\begin{frame}
\frametitle{$N^2 = -g\frac{1}{\rho}\pd{\rho}{z}$}
  \begin{block}{}
    Valores para o oceano superficial: 10--33 minutos.
  \end{block}
  \pause
  \begin{block}{}
    Valores para o oceano profundo: $\sim 6$ horas.
  \end{block}
  \pause
  \vspace{1cm}
  Obs: A frequência $\frac{N}{2\pi}$ é a frequência máxima possível para
       ondas internas.
\end{frame}


\begin{frame}
\frametitle{Variações da densidade}
\scriptsize{
  \begin{itemize}[<+-| alert@+>]
    \item A água na picnoclina é muito estável, como às turbulências tem menos
          capacidade de penetrar através dessa camada.  Formando uma ``barreira'' para movimentos verticais.
    \item Nas latitudes baixas e médias do oceano aberto, a maioria das
          variações de densidade nos 1000 m superficiais devem-se a variações
          da temperatura.
    \item A maiores profundidades, a salinidade pode desempenhar um papel
          significativo.
    \item O efeito da salinidade na camada profunda é mais evidente no
          Atlântico onde existe claramente uma estrutura estratifica da
          salinidade do que no Pacifico onde as águas profundas são mais
          uniformes.
    \item Em certos locais como o Pacífico Oriental Norte e nas regiões polares
          a salinidade influencia a densidade nas camadas superficiais.
    \item Nas águas costeiras, fiordes e estuários a salinidade é muitas vezes
          o fator dominante na determinação da densidade a todas as
          profundidades.
  \end{itemize}
  }
\end{frame}


\begin{frame}
\frametitle{Mistura de massas d'água}
  \begin{center}
    \includegraphics[scale=0.3]{../figures/water_mass_mix_01.png}
  \end{center}
\end{frame}


\section{Oceano austral}
\begin{frame}
\frametitle{Oceano austral}
  \begin{columns}
    \begin{column}{0.7\textwidth}
      \includegraphics[scale=0.3]{../figures/antartic.png}
    \end{column}

    \begin{column}{0.3\textwidth}
      \includegraphics[scale=0.15]{../figures/antartic_section.png}
    \end{column}
  \end{columns}
\end{frame}


\begin{frame}
\frametitle{Salinidade}
    \begin{center}
      \includegraphics[scale=0.35]{../figures/antartic_salinity.png}
    \end{center}
\end{frame}


\begin{frame}
\frametitle{Temperatura}
    \begin{center}
      \includegraphics[scale=0.35]{../figures/antartic_potential_temperature.png}
    \end{center}
\end{frame}


\begin{frame}
\frametitle{Sigma}
    \begin{center}
      \includegraphics[scale=0.35]{../figures/antartic_sigma.png}
    \end{center}
\end{frame}


\begin{frame}
\frametitle{Oxigênio}
    \begin{center}
      \includegraphics[scale=0.35]{../figures/antartic_oxigen.png}
    \end{center}
\end{frame}


\begin{frame}
\frametitle{Tratamento de dados}
  \begin{block}{Exercício de tratamentos de dados}
    Será fornecido um conjunto de dados hidrográficos para vocês
    trabalharem ao longo do ano.
  \end{block}

  \url{http://dl.dropbox.com/u/4411725/unimonte/OcFis/tratamento_de_dados/dados/index.html}
\end{frame}


\end{document}
