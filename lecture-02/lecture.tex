
% Title page.
\title[Aula 02]{Oceanografia Física Descritiva}
\subtitle{Dimensões dos oceanos e Propriedades físicas da água do mar}
\author[Filipe Fernandes]{Filipe P. A. Fernandes}
\institute[unimonte]{Centro Universitário Monte Serrat}
\date[Agosto 2013]{16 de Agosto 2013}

\logo{\includegraphics[scale=0.15]{../common/university_logo.png}}

\begin{document}

% The title page frame.
\begin{frame}[plain]
  \titlepage
    \begin{center}
    \includegraphics[scale=0.5]{../common/by-nc-sa.pdf}
  \end{center}
\end{frame}

\section*{Outline}
\begin{frame}
\tableofcontents
\end{frame}


\section{Propriedades físicas da água do mar}
\begin{frame}
\frametitle{Leitura}
    \begin{block}{}
        Capítulos 3.6, 6.1, 6.2, 6.5 do Stewart.

        \small{\url{http://oceanworld.tamu.edu/resources/ocng_textbook/contents.html}}
    \end{block}
\end{frame}

\begin{frame}
\frametitle{O que observamos/estimamos no oceano?}
Variáveis observadas:
    \small{
    {\bf
    \begin{itemize}[<+-| alert@+>]
    \item Temperatura
    \item Salinidade
    \item Pressão
    \item Elevação do nível do mar
    \end{itemize}
    }}
    \pause
Variáveis estimadas:
    \small{
    \begin{itemize}[<+-| alert@+>]
    \item Densidade da água do mar
    \item Densidade Potencial
    \item Velocidade do som
    \end{itemize}
    }
\end{frame}

\subsection{Temperatura}
\begin{frame}
\frametitle{Definição de temperatura}
    \small{
    \begin{block}{}
    Propriedade termodinâmica do fluido que mede a energia cinética associada
    ao movimento desordenado dos átomos e moléculas.
    \end{block}
    }
    \pause
    \begin{center}
        \includegraphics[scale=0.4]{../figures/temperatura.png}
    \end{center}
    \pause
    \footnotesize{
    \begin{itemize}[<+-| alert@+>]
    \item Quanto maior a atividade (energia), maior a temperatura;
    \item Temperatura é uma medida do calor armazenado;
    \item Unidades: \textcelsius, K.
    \end{itemize}
    }
\end{frame}

\begin{frame}
\frametitle{Como medir temperatura?}
    É a propriedade do oceano que pode ser mais precisamente medida.
    \small{
    \begin{itemize}[<+-| alert@+>]
    \item {\bf Expansão de um líquido ou metal};
    \item Diferença na expansão de dois metais;
    \item Pressão de vapor de um líquido;
    \item {\bf Variação na resistência elétrica};
    \item {\bf Radiação de microondas da superfície do mar}.
    \item Radiação infravermelho da superfície do mar;
    \end{itemize}
    }
\end{frame}

\begin{frame}
\frametitle{Como medir temperatura?}
    Termômetros de mercúrio
    \small{
    \begin{itemize}[<+-| alert@+>]
    \item Desenvolvimento do termômetro de reversão por Negretti e Zambra em
          1874;
    \item Separa o mercúrio na coluna expandida e captura a temperatura na
          profundidade da inversão;
    \item Variações na temperatura ainda causam pequenas alterações na
          quantidade de mercúrio;
    \item Estas alterações podem ser corrigidas no momento da leitura da
          temperatura no navio;
    \item Precisão: 0,01\textcelsius.
    \end{itemize}
    }
\end{frame}

\begin{frame}
\frametitle{Termômetros de mercúrio}
    \begin{columns}
        \begin{column}{0.5\textwidth}
    \begin{center}
        \shadowbox{\includegraphics[scale=0.2]{../figures/reversing_thermometer.jpg}}
    \end{center}
        \end{column}
    \begin{column}{0.5\textwidth}
    \begin{center}
        \shadowbox{\includegraphics[scale=0.4]{../figures/reversing_bottle.png}}
    \end{center}
    \end{column}
    \end{columns}
\end{frame}

\begin{frame}
\frametitle{Garrafas de reversão}
    \small{
    \begin{itemize}[<+-| alert@+>]
    \item Amostragem não uniforme com a profundidade;
    \item Efeito da pressão sobre a temperatura em função da profundidade;
    \item Estações hidrográficas com termômetros reversíveis ainda são
          considerados métodos mais confiáveis de medida do oceano!
    \end{itemize}
    }
\end{frame}

\begin{frame}
\frametitle{Como medir temperatura?}
    Termômetros de resistência
    \small{
    \begin{itemize}[<+-| alert@+>]
    \item Batitermógrafo descartável (XBT): método mais efetivo a baixo custo
          para obtenção de perfis de temperatura;
    \item Baseado na propriedade que a resistência elétrica dos metais muda
          com a temperatura;
    \item Metais mais comuns: cobre, platina e níquel. Platina é o mais
          estável e por isso os termômetros de resistência de platina
          tornaram-se padrão para as escalas internacionais de temperatura;
    \item Precisão: $\pm 0,001$\textcelsius.

    \end{itemize}
    }
\end{frame}

\begin{frame}
\frametitle{Termômetros de resistência: XBT}
    \begin{center}
        \shadowbox{\includegraphics[scale=0.35]{../figures/xbt.png}}
    \end{center}
\end{frame}

\begin{frame}
\frametitle{Como medir temperatura?}
    Perfilador CTD
    \small{
    \begin{itemize}[<+-| alert@+>]
    \item Perfilador de alta precisão de Condutividade e Temperatura em
          função da profundidade (CTD);
    \item Outros sensores podem ser acoplados como: oxigênio, pH, fluorômetro;
    \item Taxa de amostragem pode ser ajustada (12--48 Hz);
    \item Temperatura: medida por termistor;
    \item Condutividade: por indução;
    \item Pressão: cristal de quartzo.
    \end{itemize}
    }
\end{frame}

\begin{frame}
\frametitle{CTD}
    \begin{center}
        \shadowbox{\includegraphics[scale=0.15]{../figures/ctd_rosette_launch.png}}
    \end{center}
\end{frame}

\begin{frame}
\frametitle{Projeto ARGO}
    \begin{columns}
        \begin{column}{0.5\textwidth}
    \begin{center}
        \includegraphics[scale=1]{../figures/argo_map.png}
    \end{center}
        \end{column}
    \begin{column}{0.5\textwidth}
    \begin{center}
       \includegraphics[scale=1]{../figures/argo_cycle.png}
    \end{center}
    \end{column}
    \end{columns}
        \begin{center}
        \includegraphics[scale=0.12]{../figures/argo_01.png}    \end{center}
\end{frame}

\begin{frame}
\frametitle{Relação entre temperatura e calor}
    \small{
    \begin{itemize}[<+-| alert@+>]
    \item Calor: forma de energia.
    \item Unidade: Joule [J] = 1 kg m$^{-2}$ s$^{-1}$
    \item Variação de calor por unidade de tempo:
        $\frac{dQ}{dt} = [\text{W}] = 1 \text{Watt} = 1 \text{J} s^{-1}$
    \item Fluxo de calor (F):
        $\frac{\text{Energia}}{\text{area tempo}} = \frac{\text{F}}{\text{s m}^{2}} = \frac{\text{W}}{\text{m}^2}$
    \end{itemize}
    }
\end{frame}

\begin{frame}
\frametitle{Fluxo de calor pela superfície [W m$^{-2}$]}
    \begin{center}
        \includegraphics[scale=0.4]{../figures/heat_flux.png}
    \end{center}
\end{frame}

\begin{frame}
\frametitle{Temperatura média em superfície [\textcelsius{}]}
    \begin{center}
        \includegraphics[scale=0.5]{./figures/surface_temperature_woa09.png}
    \end{center}
\end{frame}

\begin{frame}
\frametitle{Temperatura em função da latitude}
    \begin{center}
        \includegraphics[scale=0.55]{./figures/latitudinal_temperature_woa09.png}
    \end{center}
\end{frame}

\begin{frame}
\frametitle{Temperatura potencial}
    \small{
    \begin{itemize}[<+-| alert@+>]
    \item Água do mar não é totalmente incompressível;
    \item Quanto mais fundo, maior o efeito do aquecimento devido à compressão
          da pressão hidrostática;
    \item Temperatura potencial: temperatura que a água teria se a movêssemos
          adiabaticamente para um outro nível de pressão;
    \end{itemize}
    }
\end{frame}

\begin{frame}
\frametitle{Temperatura potencial}
    \begin{center}
        \includegraphics[scale=0.5]{./figures/profile_temperature_woa09.png}
    \end{center}
\end{frame}

\begin{frame}
\frametitle{Temperatura potencial vs temperatura {\it in situ}}
    \begin{center}
        \includegraphics[scale=0.35]{../figures/potential_vs_insitu_temperature.png}
    \end{center}
\end{frame}

\subsection{Salinidade e condutividade}
\begin{frame}
\frametitle{Salinidade e condutividade}
    \begin{block}{}
        A água do mar é uma mistura de 96,5\% de moléculas de água pura e cerca
        de 3,5\% de outros materiais, tais como sais, gases dissolvidos,
        substâncias orgânicas e partículas não solúveis.
    \end{block}
\end{frame}

\begin{frame}
\frametitle{Água com sal!}
    \small{
    \begin{block}{}
        A água do mar contém em seu peso 3,5\% de sais, gases, substâncias
        orgânicas e material particulado.  A presença adicional dos sais
        influencia na maioria das propriedades físicas da água do mar
        (densidade, compressibilidade, ponto de congelamento, temperatura da
        densidade máxima) em algum grau, mas não são os fatores que as
        condicionam.
    \end{block}

    \pause
    \begin{block}{}
        Algumas propriedades (viscosidade, absorção de luz) não são
        significativamente afetadas pela salinidade (detalhe: o material
        dissolvido e particulado afeta a absorção de luz, e de fato, essa
        influência é usada na maioria das aplicações ópticas).

        Duas propriedades que são determinadas pela quantidade de sais na água
        são a condutividade e a pressão osmótica.
    \end{block}
    }
\end{frame}

% \begin{frame}
% \frametitle{Água com sal!}
%     \small{
%     \begin{itemize}[<+-| alert@+>]
%     \item O ponto de congelamento diminui com o aumento da pressão.  Como
%     consequência, ocorre derretimento na base das geleiras, facilitam o seu
%     deslocamento.
%     \item Ligações de hidrogênio quebram com o aumento da pressão, i.e., o gelo
%     sob pressão se torna mais plástico. Como consequência, o gelo terrestre da
%     Antártica e do Ártico fluem, quebrando pedaços na sua porção mais externa
%     que formam os icebergs.  Sem esse processo, toda a água das regiões polares
%     eventualmente se tornariam gelo.
%     \end{itemize}
%     }
% \end{frame}

\begin{frame}
\frametitle{Definição de salinidade}
    \small{
    \begin{block}{}
    Quantidade de matéria (inorgânica) dissolvida (expressa em gramas) por
    quilograma de água do mar.
    \end{block}
    }
    \pause
    \small{
    \begin{itemize}[<+-| alert@+>]
    \item Já foi expressa em partes por mil (\textperthousand) e por UPS
    (unidade prática de salinidade).  O padrão hoje preferido pela UNESCO é
    expressar salinidade sem unidades pois representa massa/massa [{\bf !}].
    \end{itemize}
    }
\end{frame}

\begin{frame}
\frametitle{Como medir salinidade}
    \small{
    \begin{itemize}[<+-| alert@+>]
    \item Evaporar a água do mar e pesar o resíduo (método primitivo).
    \item Determinar a quantidade de cloro, bromo e iodo para medir a
          clorinidade através de titulação com sulfato de prata (método
          antigo).
    \item Relação salinidade e clorinidade: S = 0.03 + 1,80655 $\times$ [Cl].
    \item Precisão de 0,025.
    \item Este método foi utilizado até o Ano Internacional Geofísico, 1957.
    \item Medida da condutividade (método moderno).
    \end{itemize}
    }
\end{frame}

\begin{frame}
\frametitle{Condutividade}
    \small{
    \begin{itemize}[<+-| alert@+>]
    \item A condutividade da água do mar depende da temperatura e salinidade.
          Através da medida precisa da temperatura, a salinidade pode ser
          determinada.
    \item Precisão: 0,001.
    \item A precisão da determinação da salinidade depende da precisão da água
          do mar utilizada para calibrar os instrumentos que medem a condutividade.
    \item Como medir a salinidade a partir da condutividade?
    \end{itemize}
    }
\end{frame}

\begin{frame}
\frametitle{Salinidade média em superfície [g kg$^{-1}$]}
    \begin{center}
        \includegraphics[scale=0.5]{./figures/surface_salinity_woa09.png}
    \end{center}
\end{frame}

\begin{frame}
\frametitle{Salinidade no Atlântico}
    \begin{center}
        \includegraphics[scale=0.4]{./figures/cross_section_salinity_woa09.png}
    \end{center}
\end{frame}

\begin{frame}
\frametitle{Precipitação menos Evaporação}
    \begin{center}
        \includegraphics[scale=0.35]{../figures/evaporation_precipitation.png}
    \end{center}
\end{frame}

\begin{frame}
\frametitle{Instrumentos}
    \begin{center}
        \includegraphics[scale=0.4]{../figures/salinity_instruments.png}
    \end{center}
\end{frame}

\begin{frame}
\frametitle{Medição do nível médio do mar}
Variação no oceano:
    \small{
    \begin{itemize}[<+-| alert@+>]
    \item Uma das formas mais antigas de observação oceânica;
    \item Melhor informação sobre mudanças climáticas globais.
    \item Causas da variabilidade:
        \begin{enumerate}[<+-| alert@+>]
            \item curto--termo: ondas e marés; mudanças na pressão atmosférica,
                  descarga de rios, flutuações do vento no oceano;
            \item longo--termo: estérico (variações na densidade) ou eustático
                  (variação na massa);
            \item subsidência costeira.
        \end{enumerate}
    \end{itemize}
    }
\end{frame}

\begin{frame}
\frametitle{Medição do nível médio do mar}
    \small{
    \begin{itemize}
    \item Observações de longo--termo na altura refletem variações da:
        \begin{enumerate}[<+-| alert@+>]
            \item Circulação oceânica de larga–escala;
            \item Tensão de cisalhamento do vento;
            \item Volume dos oceanos.
        \end{enumerate}
    \end{itemize}
    }
\end{frame}

\begin{frame}
\frametitle{Marégrafos de boia}
    \begin{columns}
        \begin{column}{0.5\textwidth}
        \small{
        \begin{itemize}[<+-| alert@+>]
            \item Marégrafos de boia e contrapeso: próprias para instalação em
                  estruturas fixas (piers);
            \item Mede deslocamento vertical de uma boia colocada na superfície
                  do mar;
            \item Sistema de cabo e contrapeso acopla boia à polia ligada a
                  um registrador mecânico ou eletrônico.
        \end{itemize}
        }
        \end{column}
    \begin{column}{0.5\textwidth}
    \begin{center}
        \includegraphics[scale=0.4]{../figures/maregrafo.png}
    \end{center}
    \end{column}
    \end{columns}
\end{frame}

\begin{frame}
\frametitle{Altimetria por Satélite}
    \begin{center}
       \includegraphics[scale=0.35]{../figures/altimetria.png}
    \end{center}
\end{frame}

\subsection{Pressão}
\begin{frame}
\frametitle{Pressão e profundidade}
Variação no oceano:
    \small{
    \begin{itemize}[<+-| alert@+>]
    \item Profundidade: 0 -- 11000 m
    \item Pressão: 0 -- 11000 dbar
    \end{itemize}
    }
    \pause
    Mas o que é pressão?
    \pause
    \begin{block}{}
        Pressão: força normal por unidade de área exercida pela água.

        Unidade: Pascal $\rightarrow$ 1 Pascal = 1 N m$^{-2}$
    \end{block}

    \begin{block}{}
        A pressão atmosférica é usualmente medida em bar:
        1 bar = 106 dinas cm$^{-2}$ = 105 N m$^{-2}$

        A pressão no oceano é usualmente medida em dbar:
        1 dbar = 10$^{-1}$ bar = 105 dinas cm$^{-2}$ = 104 Pascal.
    \end{block}
\end{frame}

\begin{frame}
\frametitle{Pressão X profundidade}
    \begin{block}{}
    A medida de profundidade é baseada na medida da pressão hidrostática.
    Isto é possível devido a relação quase linear entre $p = p(z)$ e $z$;
    \end{block}

    \pause
    \begin{block}{}
    Abaixo de 4000 m, a diferença é menor que 2\%.
    $p(z) = -g\int^0_z\rho(z)dz,$

    $g = 9,81$ m s$^{-2}$,

    $\rho = 1,025 \times 10^3$ kg m$^{-3}$

    $\rho = 1025 (9,81)z = 1,005525z$ [dbar].
    \end{block}
\end{frame}

\begin{frame}
\frametitle{Pressão hidrostática}
    \begin{columns}
        \begin{column}{0.7\textwidth}
        \small{
        \begin{itemize}[<+-| alert@+>]
            \item Técnicas de medição: perfil contínuo (CTD) ou determinação
                  em profundidades discretas;
            \item Discreta: Determinação da pressão termométrica combinando
                  termômetro protegido e desprotegido para sentir os efeitos
                  da pressão na temperatura;
            \item Considerado o método mais preciso para se determinar a
                  pressão hidrostática;
            \item Usado para calibrar perfis de CTD.
        \end{itemize}
        }
        \end{column}
    \begin{column}{0.3\textwidth}
    \begin{center}
        \shadowbox{\includegraphics[scale=0.15]{../figures/reversing_thermometer.jpg}}
    \end{center}
    \end{column}
    \end{columns}
\end{frame}

\begin{frame}
\frametitle{Profundidade por velocidade de queda}
        \small{
        \begin{itemize}[<+-| alert@+>]
            \item Batitermógrafo descartável (XBT): método mais efetivo a baixo
                  custo para obtenção de perfis de temperatura;
            \item XBTs não medem profundidade diretamente mas a inferem pelo
                  ``tempo de queda'' do instrumento;
            \item Esta pode ser uma possível fonte de erro nas medidas;
            \item Erros na profundidade do XBT foram primeiramente reportados
                  por Flierl e Robinson (1977).
        \end{itemize}
        }
    \pause
    \begin{block}{}
    Thadathil, Pankajakshan, A. K. Saran, V. V. Gopalakrishna, P. Vethamony, Nilesh Araligidad, Rick Bailey, 2002: XBT Fall Rate in Waters of Extreme Temperature: A Case Study in the Antarctic Ocean. J. Atmos. Oceanic Technol., 19, 391–396.
    \end{block}
\end{frame}


\begin{frame}
\frametitle{Profundidade por velocidade de queda}
    \begin{center}
        \shadowbox{\includegraphics[scale=0.35]{../figures/xbt.png}}
    \end{center}
\end{frame}


\subsection{Densidade}
\begin{frame}
\frametitle{Densidade da água do mar}
    \begin{block}{}
    \begin{center}
        $\rho = \rho(S, T, p)$
    \end{center}
    \end{block}
    \pause
    \footnotesize{
    \begin{itemize}[<+-| alert@+>]
    \item Unidade: massa/volume (kg m$^-3$)
    \item Densidade máxima da água pura ocorre a 4\textcelsius, na pressão
          atmosférica:\\
             $\rho$(0, 4\textcelsius, 1 bar) = 1000 kg m$^{-3}$ = 1 g cm$^{-3}$
    \item Densidade da água do mar: 1025 kg m na superfície até 1050 kg m$^{-3}$
          no fundo do mar, devido à compressão.
    \item $\sigma = \rho -1000$ kg m$^{-3}$
    \item Importante: a dependência da densidade com T e S é não--linear.
    \item Calculada a partir da equação de estado da água do mar.
    \end{itemize}
    }
\end{frame}


\begin{frame}
\frametitle{Relações de TS com a densidade}
    \begin{center}
        \includegraphics[scale=0.42]{../figures/densidade.png}
    \end{center}
\end{frame}

\begin{frame}
\frametitle{\small{Isopicnais -- derivada da equação de estado da água do mar.}}
    \begin{center}
        \includegraphics[scale=0.4]{../figures/equation_of_state.png}
    \end{center}
\end{frame}


\begin{frame}
\frametitle{Para pensar...}
    \footnotesize{
    \begin{itemize}[<+-| alert@+>]
    \item O ponto de congelamento da água do mar diminui conforme aumenta a
          salinidade.
    \item A temperatura da densidade máxima diminui conforme a
          salinidade aumenta.
    \item Formação de gelo marinho:
    \url{http://www.bbc.co.uk/nature/15835017}
    \end{itemize}
    }
\end{frame}


\begin{frame}
\frametitle{Efeito da pressão na densidade: Densidade Potencial}
    \footnotesize{
    \begin{itemize}[<+-| alert@+>]
    \item A água do mar é compressível, embora não tanto quanto o gás.
    \item Maiores variações na água do mar são causadas pela variação da
          pressão.
    \item Densidade potencial: densidade de uma parcela quando deslocada
          adiabaticamente a uma pressão de referência: ex.: superfície.
    \item $\sigma_t = \sigma(S, T, 0)$
    \item Água mais fria é mais compressível que água mais morna.
    \item Portanto, água fria se torna mais densa que água morna quando
          submergem à mesma pressão. Exemplo: água do Mediterrâneo e da
          Groenlândia.
    \end{itemize}
    }
\end{frame}


\begin{frame}
\frametitle{Efeitos de TS na compressibilidade}
  \begin{columns}
    \begin{column}{0.5\textwidth}
      \begin{center}
        \includegraphics[scale=0.4]{../figures/compressibilidade_densidade.png}
      \end{center}
    \end{column}

    \begin{column}{0.5\textwidth}
      \small{
      \begin{itemize}[<+-| alert@+>]
        \item Parcela mais quente e salgada referenciada na superfície é mais
              densa que a parcela mais fria e doce;
        \item A 4000 dbar, a parcela quente e salgada é mais leve que a fria e
              doce.
      \end{itemize}
      }
    \end{column}
  \end{columns}
\end{frame}


\subsection{Som}
\begin{frame}
\frametitle{Som na água}
    \begin{block}{}
        Mas porque estudar o som?
    \end{block}
\end{frame}


\begin{frame}
\frametitle{Som na água}
    \footnotesize{
    \begin{itemize}
    \item O som se propaga como uma onda de pressão;
    \item Pode transmitir informação sobre grandes distâncias no oceano;
    \item Usado para medir propriedades do fundo do oceano, profundidade do
          oceano, temperatura e correntes. (Baleias utilizam o som para
          navegar, comunicar a grandes distâncias e procurar comida.)
    \item A velocidade do som no oceano varia com a temperatura, salinidade e
          pressão.
    \item Valores típicos: 1450 m s$^{-1}$ a 1550 m s$^{-1}$
    \end{itemize}
    }
\end{frame}


\begin{frame}
\frametitle{Variação do som com T e S}
    \begin{center}
        \includegraphics[scale=0.4]{../figures/som.png}
    \end{center}
\end{frame}


\begin{frame}
\frametitle{Canal do som}
  \begin{block}{}
    O som se propaga por grandes distâncias no canal.
  \end{block}

    \begin{center}
        \includegraphics[scale=0.7]{../figures/sound_channel.png}
    \end{center}
\end{frame}


\begin{frame}
\frametitle{Dever de casa 01}
  \begin{block}{}
    Plotar os perfis de temperatura, salinidade dos dados fornecidos*.

    Discutir as propriedades que eu deixei de fora: Luz, nutrientes e etc

    Desafio: Plotar um diagrama TS.
    \end{block}

    \pause
    \begin{block}{}
        * e-mail? Dropbox?
    \end{block}
\end{frame}


\begin{frame}
\frametitle{Aula de reposição...}
  \begin{block}{}
    ???
  \end{block}
\end{frame}

\end{document}
