% % Title page.
% \title[Aula 10]{Correntes oceânicas}
% \section{Circulação termohalina}
% \subsection{Leis de conservação}
% \subsubsection{Conservação de massa}
% \begin{frame}
% \frametitle{Conservação de massa}
% {\small
%   \begin{itemize}[<+-| alert@+>]
%     \item A lei de conservação da massa é utilizada em Oceanografia em diversas
%           situações;
%     \item Por exemplo, para um canal, esta lei exige que a massa de água que
%           passa pela entrada menos a massa que passa pela saída do canal seja
%           igual à variação da massa de água no interior do canal (normalmente
%           indicada pela variação do nível da superfície da água).
%     \item E no meio oceânico como um todo, a massa de água advinda de
%           precipitações de chuva e neve mais a recebida por rios e derretimento
%           de gelo é aproximadamente igual à massa de água perdida por
%           evaporação mais a que congela.
%   \end{itemize}
% }
% \end{frame}
%
% \subsubsection{Ciclo hidrológico/Conservação do volume}
% \begin{frame}
% \frametitle{Conservação do volume}
%   \begin{itemize}[<+-| alert@+>]
%     \item Princípio da conservação do volume (ou equação da continuidade):
%     \item Consequência do fato da compressibilidade da água ser pequena.
%   \end{itemize}
%   \begin{block}{}
%   \[
%     V_i + R + P = V_o + E
%   \]
%   \[
%     Vo - V_i = R + P - E
%   \]
%   \end{block}
% \end{frame}
%
%
% \begin{frame}
% \frametitle{Equação da continuidade}
% \[
%   \nabla_h . \mathbf{u} + \pd{w}{z} = 0
% \]
%
% ou,
%
% \[
%   \pd{u}{x} + \pd{v}{y} + \pd{w}{z} = 0
% \]
% \end{frame}
%
%
% \subsubsection{Conservação do sal}
% \begin{frame}
% \frametitle{Conservação de salinidade}
% \[
%   \pd{S}{t} + (\mathbf{u} . \nabla_h)S + w\pd{S}{z} = \pd{}{z}\left( K_H\pd{S}{z} \right) + F_S
% \]
%
% ou (modelo de caixa),
% \begin{block}{}
% \[
%   V_iS_i = V_oS_o
% \]
% \end{block}
% \end{frame}
%
%
% \subsubsection{Conservação de calor}
% \begin{frame}
%   \frametitle{Conservação de calor}
%   \begin{itemize}[<+-| alert@+>]
%   \item A temperatura das águas oceânicas varia no espaço e no tempo.
%   \item Esta variação depende dos fluxos de calor que entram e saem em cada
%         corpo d'água, e o cálculo desses fluxos se denomina "balanço térmico".
%   \item Esses fluxos de radiação normalmente são denotados pela letra Q,
%         com índices identificando cada componente do balanço térmico.
%   \item Símbolo Q: fluxo: calor por unidade de tempo por unidade de área
%   \end{itemize}
% \end{frame}
%
% \begin{frame}
% \frametitle{Conservação de temperatura}
% \[
%   \pd{\theta}{t} + (\mathbf{u} . \nabla_h)\theta + w\pd{\theta}{z} = \pd{}{z}\left( K_H\pd{\theta}{z} \right) + F_{\theta}
% \]
% \end{frame}
%
% \begin{frame}
%   \frametitle{Conservação de calor}
%   \begin{itemize}[<+-| alert@+>]
%   \item $Q_S \rightarrow$ Radiação de onda curta (solar incidente);
%   \item $Q_B \rightarrow$ Radiação de onda longa (emitida pelo oceano e
%                           refletida por nuvens);
%   \item $Q_H \rightarrow$ Calor por condução (Sensível);
%   \item $Q_E \rightarrow$ Calor por evaporação/condensação (Latente);
%   \item $Q_V \rightarrow$ Calor por advecção.
%   \end{itemize}
% \end{frame}
%
% \begin{frame}
%   \frametitle{Conservação de calor}
%   \begin{block}{}
%     \[
%       Q_{net} = Q_S + Q_V + Q_B + Q_H + Q_E + Q_T
%     \]
%   \end{block}
% \end{frame}
%
% \subsection{Teoria da radiação eletromagnética}
% \begin{frame}
%   \frametitle{Teoria da radiação eletromagnética}
%   Lei de Stefan diz que um corpo com temperatura $T$ emite radiação  segundo:
%   \begin{block}{}
%     \[
%       Q \sim \sigma T^4
%     \]
%   \end{block}
%   onde, $\sigma \approx 5.7338$ J m$^{-2}$\textdegree{K}$^{-4}$s$^{-1}$
% \end{frame}
%
% \subsubsection{Radiação de ondas curtas}
% \begin{frame}
%   \frametitle{Ondas curtas}
%   \begin{itemize}[<+-| alert@+>]
%     \item 48\% atinge o mar, dos quais
%     \item 29\% é radiação direta (do sol)
%     \item 19\% é radiação indireta (após espalhamento na atmosfera)
%   \end{itemize}
% \end{frame}
%
% \begin{frame}
%   \frametitle{Fatore que afetam $Q_S$}
%   \begin{itemize}[<+-| alert@+>]
%     \item Duração do dia (depende da estação do ano e da latitude).
%     \item Elevação do sol (Densidade de energia é proporcional ao seno do
%           ângulo de elevação do sol).
%     \item Absorção na atmosfera (devido a moléculas de gás, poeira , vapor
%           d'água, ...)
%     \item Efeito de nuvens: que diminui a radiação direta e daí a total,
%           embora aumente a radiação indireta.
%     \item Efeito de ondas na superfície do mar. (que modificam os ângulos de
%           incidência dos raios de sol).
%   \end{itemize}
% \end{frame}
%
% \subsubsection{Radiação de ondas longas}
% \begin{frame}
%   \frametitle{Ondas longas}
%   \begin{block}{}
%     $Q_B$ proporcional à temperatura absoluta da superfície elevado à
%           quarta potência.
%   \end{block}
% \end{frame}
%
% \begin{frame}
%   \frametitle{Fatores que influenciam $Q_B$}
%   {\footnotesize
%   \begin{itemize}[<+-| alert@+>]
%     \item Espessura das nuvens. Quanto maior a espessura da nuvem, menos é o
%           calor que escapa para o espaço.
%     \item A altura das nuvens, o que determina a temperatura em que a nuvem
%           irradia calor de volta ao oceano. Nuvens altas são mais frias do que
%           as nuvens baixas.
%     \item Quantidade de vapor atmosférico (quanto mais úmido, menor a quantidade
%           de calor que escapa para o espaço).
%     \item Temperatura da água (quanto maior a temperatura maior a radiação).
%     \item Cobertura de gelo e neve. O gelo emite como um corpo negro, mas esfria
%           muito mais rápido do que as áreas de oceano aberto.  Mares cobertos de
%           gelo são isolados da atmosfera.
%   \end{itemize}
% }
% \end{frame}
%
% \subsubsection{Evaporação}
% \begin{frame}
%   \frametitle{Fatores que influenciam $Q_E$}
%   \begin{itemize}
%     \item Vento (aumento do vento aumenta a evaporação).
%     \item Umidade relativa do ar (ar seco aumenta a evaporação).
%   \end{itemize}
% \end{frame}
%
% \begin{frame}
%   \frametitle{Evaporação}
%   \begin{block}{}
%     Evaporação é muito importante, mas difícil de determinar diretamente.
%     Implica em perda de volume d'água e de calor.
%   \end{block}
% \end{frame}
%
% \subsubsection{Condução de calor}
% \begin{frame}
%   \frametitle{Condução}
%   \begin{block}{}
%     Altamente dependente da diferença de temperatura entre o oceano e a atmosfera.
%   \end{block}
%
% \end{frame}
%
% \begin{frame}
%   \frametitle{$Q_{net}$}
%   \begin{center}
%     \includegraphics[scale=0.38]{../figures/convex_narr_fluxes.png}
%   \end{center}
% \end{frame}
%
%
% \begin{frame}
%   \frametitle{Distribuição meridional}
%   \begin{center}
%     \includegraphics[scale=0.48]{../figures/Qnet.png}
%   \end{center}
% \end{frame}
%
%
% \subsubsection{Razão de Bowen}
% \begin{frame}
%   \frametitle{Razão de Bowen}
%   \begin{block}{}
%     \[
%       R = \frac{Q_H}{Q_E}
%     \]
%
%     \[
%       R = 0.062 \frac{(t_s - t_a)}{e_s - e_a}
%     \]
%   \end{block}
% \end{frame}

\author[Filipe Fernandes]{Filipe P. A. Fernandes}
\institute[unimonte]{Centro Universitário Monte Serrat}
\date[Novembro 2012]{12 de Novembro 2012}

\logo{\includegraphics[scale=0.15]{../common/university_logo.png}}

\begin{document}

% The title page frame.
\begin{frame}[plain]
  \titlepage
\end{frame}

\section*{Outline}
\begin{frame}
\tableofcontents
\end{frame}
%
% Circulação e massas d'água dos oceanos.
% 6 Circulação e massas d'água dos oceanos
% 6.1 Aspectos gerais da circulação oceânica
% 6.2 Convergência e divergência de massas d'água.
% Ekman
% 6.3 Padrões de circulação nos oceanos: A circulação de superfície
% 6.3.1 A circulação de superfície no Oceano Atlântico
% 6.4 Massas d'água nos oceanos
% 6.4.1 Massas d'água do Oceano Atlântico

\section{Correntes Oceânicas}
\begin{frame}
\frametitle{Correntes Oceânicas}
  \begin{itemize}[<+-| alert@+>]
    \item Uma corrente é caracterizada por um fluxo de água no oceano que
          apresenta uma distribuição coerente em termos de médias temporais;
    \item A importância de uma corrente é avaliada pelo seu transporte (tanto
          de volume como de calor) e pela variabilidade dos mesmos;
    \item As correntes oceânicas tem uma contribuição extremamente relevante no
          transporte de calor para os polos (principalmente em latitudes médias);
  \end{itemize}
\end{frame}


\begin{frame}
\frametitle{Correntes Oceânicas}
  \begin{itemize}[<+-| alert@+>]
    \item O estudo da circulação oceânica pode ocorrer através de observações
          in situ (navios, boias, instrumentos fundeados, satélites,
          derivadores), modelos analíticos e modelos numéricos.
  \end{itemize}
\end{frame}


\begin{frame}
\frametitle{Como são geradas}
  \begin{itemize}[<+-| alert@+>]
    \item As correntes oceânicas são geradas por dois mecanismos:
    \item {\bf Circulação gerada pela vento (0-1000 m):}
    \item Associada ao padrões de distribuição de ventos globais que formam os
          giros oceânicos em escalas de bacias;
    \item Processos desde variação sazonal até escalas climáticas;
    \item Escala de bacias.
   \end{itemize}
\end{frame}


\begin{frame}
\frametitle{Como são geradas}
  \begin{itemize}[<+-| alert@+>]
    \item {\bf Circulação termo-halina (todo o oceano):}
    \item Processos relacionados as trocas de calor (aquecimento, resfriamento)
          e ou água doce (evaporação, precipitação).
    \item Processos em escalas climáticas;
    \item Escala global.
   \end{itemize}
\end{frame}

\begin{frame}
  \frametitle{Escalas espaço-temporais}
  Atmosfera
  {\scriptsize
\begin{table}
    \begin{tabular}{|l|l|l|l|}
        \hline
        Fenômeno                              & Escala espacial    & Escala temporal    &  \\ \hline
        vórtices turb.                        & poucos metros      & segundos a minutos & A   \\
        thunderstorms e tornados              & 10 m até 10 km     & minutos até horas  & A/B \\
        brisa mar/terrestre brisa vales/mont. & 5 km até 100 km    & horas até dias     & B   \\
        furacões e ciclones tropicais         & 100 km até 500 km  & dias até semana    & C   \\
        frentes atmosf.                       & 100 km até 5000 km & semanal            & C   \\
        ventos pred.                          & global             & sazonal até anual  & D   \\
        variações clim                        & global             & decadal            & D   \\
        \hline
    \end{tabular}
\end{table}
A -- micro-escala (comprimento típico de 2 m)\\
B -- meso-escala (comprimento típico de 20 km)\\
C -- escala sinótica (comprimento típico de 2000 km)\\
D -- escala global (comprimento típico de 5000 km)\\
}
\end{frame}

\begin{frame}
  \frametitle{Escalas espaço-temporais}
  Oceano
  {\scriptsize
\begin{table}
    \begin{tabular}{|l|l|l|}
        \hline
        Fenômeno                   & Escala espacial     & Escala temporal   \\ \hline
        ondas de grav. superficial & 10 cm até 100 m     & segundos          \\
        ondas internas             & 1 m até 1 km        & minutos até dias  \\
        marés                      & 100 km até 10000 km & dia               \\
        processos costeiros        & 1 km até 100 km     & vários dias       \\
        vórtices e frentes         & 10 km até 1000 km   & dias até semanas  \\
        correntes                  & 50 km até 500 km    & semanal a sazonal \\
        giros oceânicos            & escala de bacia     & anos\\
        \hline
    \end{tabular}
\end{table}
}
\end{frame}

\begin{frame}
  \frametitle{Agentes Forçantes}
  \begin{itemize}[<+-| alert@+>]
    \item[1] Vento.
    \item[2] Os fluxos entre o oceano e a atmosfera:
    \begin{enumerate}[<+-| alert@+>]
      \item Fluxos de calor (balanço de radiação, trocas de calor latente e
            calor sensível)
      \item Fluxo de água doce (precipitação e evaporação)
    \end{enumerate}
  \end{itemize}

  \pause
  \begin{block}{}
  O efeito dos fluxos no oceano:\\
  Resfriamento e evaporação $\rightarrow$ densidade aumenta\\
  Aquecimento e precipitação $\rightarrow$ densidade diminui
  \end{block}

\end{frame}


\begin{frame}
  \frametitle{Circulação do vento -- Modelo 00}
  \begin{center}
    \includegraphics[scale=0.6]{../figures/vento_simples.png}
  \end{center}
\end{frame}


\begin{frame}
  \frametitle{Circulação do vento -- Modelo 01}
  \begin{center}
    \includegraphics[scale=0.6]{../figures/vento_simples_02.png}
  \end{center}
\end{frame}


\subsection{Tensão de cisalhamento do vento e a camada de Ekman}
\begin{frame}
  \frametitle{Circulação do vento (Células de vento)}
  \begin{center}
    \includegraphics[scale=0.45]{../figures/wind_circulation.png}
  \end{center}
\end{frame}


\begin{frame}
  \frametitle{Tensão de cisalhamento do vento}
  \begin{center}
    \includegraphics[scale=0.38]{../figures/wind_stress.png}
  \end{center}
  \[
    (\tau_{wind_x}, \tau_{wind_y}) = \rho_{air} C_D U_{10}(u_a, v_a)
  \]
\end{frame}


\begin{frame}
  \frametitle{Relação entre tensão de cisalhamento do vento e forças de fricção.}
  \begin{block}{}
    Tensão de cisalhamento = -fluxo de momento
  \end{block}
  \begin{center}
    \includegraphics[scale=0.27]{../figures/friction_forces.png}
  \end{center}
\end{frame}


\begin{frame}
  \frametitle{Turbulência}
  \begin{center}
    \includegraphics[scale=0.31]{../figures/turbulence.png}
  \end{center}
\end{frame}


\subsection{Ekman}
\begin{frame}
  \frametitle{Balanço da força de Ekman}
  \begin{columns}
    \begin{column}{0.5\textwidth}
      \begin{center}
        \includegraphics[scale=0.25]{../figures/ekman_scheme.png}
      \end{center}
    \end{column}
    \begin{column}{0.5\textwidth}
      \[
        -fv_e = F_x = \frac{1}{\rho_{ref}}\pd{\tau_x}{z}
      \]
      \[
        fu_e = F_y = \frac{1}{\rho_{ref}}\pd{\tau_y}{z}
      \]
    \end{column}  \end{columns}
\end{frame}


\begin{frame}
  \frametitle{Transporte de Ekman}
  \begin{columns}
    \begin{column}{0.5\textwidth}
      \begin{center}
        \includegraphics[scale=0.25]{../figures/spiral_transport.png}
      \end{center}
    \end{column}
    \begin{column}{0.5\textwidth}
      \[
        \mathbf{M}_e = \int_{-D}^0 \rho_{ref}\mathbf{u}_edz
      \]
      \[
        \mathbf{M}_e = \frac{\tau_{wind} \times \hat{z}}{f}
      \]
    \end{column}  \end{columns}
\end{frame}


\begin{frame}
  \frametitle{Bombeamento (e sucção) de Ekman}
  Convergência e divergência no transporte de Ekman causa movimentos verticais.
  \begin{block}{}
    \[
      \pd{w}{z} = -\nabla_h . \mathbf{u}_e \text{, Assume }w=0 \text{ em } z=0
    \]
    \[
      w_{ek} = -\frac{1}{\rho_{ref}}\nabla_h . \mathbf{M}_e \text{, ou}
    \]
    \[
      w_{ek} = -\frac{1}{\rho_{ref}}\hat{z} . \nabla \times \left( \frac{\tau_{wind}}{f} \right)
    \]
  \end{block}
\end{frame}

\begin{frame}
  \frametitle{Distribuição do bombeamento de Ekman}
      \begin{center}
        \includegraphics[scale=0.42]{../figures/ekman_pump.png}
      \end{center}
\end{frame}


\subsection{Balanço Geostrófico}
\begin{frame}
  \frametitle{Balanço Geostrófico}
  {\scriptsize
  \begin{itemize}[<+-| alert@+>]
    \item Ocorre amplamente no interior do oceano e da atmosfera;
    \item É representado pelo equilíbrio entre a Força do Gradiente de Pressão
          e a Força de Coriolis;
    \item É um equilíbrio estacionário;
    \item A pressão em um ponto no interior do oceano é função do peso de água
          acima deste, que é função da densidade $\rho(S, T, P)$ e da altura da
          coluna d'água.
  \end{itemize}
  \pause
  \begin{block}{}
    \[
      u = -\frac{1}{f\rho}\pd{p}{y}
    \]

    \[
      v = -\frac{1}{f\rho}\pd{p}{x}
    \]
  \end{block}
}
\end{frame}


\begin{frame}
  \frametitle{Correntes}
  \begin{center}
    \includegraphics[scale=1.3]{../figures/currents_cartoon.png}
  \end{center}
\end{frame}

\subsection{Sumário}
\begin{frame}
\frametitle{Sumário}
  \begin{itemize}[<+-| alert@+>]
    \item A circulação gerada pelo vento (principalmente associada aos giros
          subtropicais), apresenta uma forte simetria entre o HN e HS.
    \item A circulação envolvendo os 1000 m do oceano é mais rápida, ocorre em
          escalas de bacias e medidas diretas podem ser usadas para a sua
          avaliação.
    \item A circulação termohalina é lenta, ocorre em escala global e é
          geralmente estudada através do conceito de massas de água.
  \end{itemize}
\end{frame}


\begin{frame}
  \frametitle{Desenhar o balanço de forcas!}
\end{frame}

\begin{frame}
  \frametitle{Dever de casa}
  \pause
    Chequem o portal!
\end{frame}

\end{document}
