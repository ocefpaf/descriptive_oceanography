% Title page.
\title[Aula 10]{Correntes oceânicas}
\author[Filipe Fernandes]{Filipe P. A. Fernandes}
\institute[unimonte]{Centro Universitário Monte Serrat}
\date[Novembro 2013]{08 de Novembro 2013}

\logo{\includegraphics[scale=0.15]{../common/university_logo.png}}

\begin{document}

% The title page frame.
\begin{frame}[plain]
  \titlepage
\end{frame}

\section*{Outline}
\begin{frame}
\tableofcontents
\end{frame}

\section{Correntes Oceânicas}
\begin{frame}
\frametitle{Correntes Oceânicas}
  \begin{itemize}[<+-| alert@+>]
    \item Uma corrente é caracterizada por um fluxo de água no oceano que
          apresenta uma distribuição coerente em termos de médias temporais;
    \item A importância de uma corrente é avaliada pelo seu transporte (tanto
          de volume como de calor) e pela variabilidade dos mesmos;
    \item As correntes oceânicas tem uma contribuição extremamente relevante no
          transporte de calor para os polos (principalmente em latitudes médias);
  \end{itemize}
\end{frame}


\begin{frame}
\frametitle{Correntes Oceânicas}
  \begin{itemize}[<+-| alert@+>]
    \item O estudo da circulação oceânica pode ocorrer através de observações
          in situ (navios, boias, instrumentos fundeados, satélites,
          derivadores), modelos analíticos e modelos numéricos.
  \end{itemize}
\end{frame}


\begin{frame}
\frametitle{Como são geradas}
  \begin{itemize}[<+-| alert@+>]
    \item As correntes oceânicas são geradas por dois mecanismos:
    \item {\bf Circulação gerada pela vento (0-1000 m):}
    \item Associada ao padrões de distribuição de ventos globais que formam os
          giros oceânicos em escalas de bacias;
    \item Processos desde variação sazonal até escalas climáticas;
    \item Escala de bacias.
   \end{itemize}
\end{frame}


\begin{frame}
\frametitle{Como são geradas}
  \begin{itemize}[<+-| alert@+>]
    \item {\bf Circulação termohalina (todo o oceano):}
    \item Processos relacionados as trocas de calor (aquecimento, resfriamento)
          e ou água doce (evaporação, precipitação).
    \item Processos em escalas climáticas;
    \item Escala global.
   \end{itemize}
\end{frame}

% \begin{frame}
%   \frametitle{Escalas espaço-temporais}
%   Atmosfera
%   {\scriptsize
% \begin{table}
%     \begin{tabular}{|l|l|l|l|}
%         \hline
%         Fenômeno                              & Escala espacial    & Escala temporal    &  \\ \hline
%         vórtices turb.                        & poucos metros      & segundos a minutos & A   \\
%         thunderstorms e tornados              & 10 m até 10 km     & minutos até horas  & A/B \\
%         brisa mar/terrestre brisa vales/mont. & 5 km até 100 km    & horas até dias     & B   \\
%         furacões e ciclones tropicais         & 100 km até 500 km  & dias até semana    & C   \\
%         frentes atmosf.                       & 100 km até 5000 km & semanal            & C   \\
%         ventos pred.                          & global             & sazonal até anual  & D   \\
%         variações clim                        & global             & decadal            & D   \\
%         \hline
%     \end{tabular}
% \end{table}
% A -- micro-escala (comprimento típico de 2 m)\\
% B -- meso-escala (comprimento típico de 20 km)\\
% C -- escala sinótica (comprimento típico de 2000 km)\\
% D -- escala global (comprimento típico de 5000 km)\\
% }
% \end{frame}

% \begin{frame}
%   \frametitle{Escalas espaço-temporais}
%   Oceano
%   {\scriptsize
% \begin{table}
%     \begin{tabular}{|l|l|l|}
%         \hline
%         Fenômeno                   & Escala espacial     & Escala temporal   \\ \hline
%         ondas de grav. superficial & 10 cm até 100 m     & segundos          \\
%         ondas internas             & 1 m até 1 km        & minutos até dias  \\
%         marés                      & 100 km até 10000 km & dia               \\
%         processos costeiros        & 1 km até 100 km     & vários dias       \\
%         vórtices e frentes         & 10 km até 1000 km   & dias até semanas  \\
%         correntes                  & 50 km até 500 km    & semanal a sazonal \\
%         giros oceânicos            & escala de bacia     & anos\\
%         \hline
%     \end{tabular}
% \end{table}
% }
% \end{frame}

\begin{frame}
  \frametitle{Agentes Forçantes}
  \begin{itemize}[<+-| alert@+>]
    \item[1] Vento.
    \item[2] Os fluxos entre o oceano e a atmosfera:
    \begin{enumerate}[<+-| alert@+>]
      \item Fluxos de calor (balanço de radiação, trocas de calor latente e
            calor sensível)
      \item Fluxo de água doce (precipitação e evaporação)
    \end{enumerate}
  \end{itemize}

  \pause
  \begin{block}{}
  O efeito dos fluxos no oceano:\\
  Resfriamento e evaporação $\rightarrow$ densidade aumenta\\
  Aquecimento e precipitação $\rightarrow$ densidade diminui
  \end{block}

\end{frame}


\begin{frame}
  \frametitle{Circulação do vento -- Modelo 00}
  \begin{center}
    \includegraphics[scale=0.6]{../figures/vento_simples.png}
  \end{center}
\end{frame}


\begin{frame}
  \frametitle{Circulação do vento -- Modelo 01}
  \begin{center}
    \includegraphics[scale=0.6]{../figures/vento_simples_02.png}
  \end{center}
\end{frame}


\subsection{Tensão de cisalhamento do vento e a camada de Ekman}
\begin{frame}
  \frametitle{Circulação do vento (Células de vento)}
  \begin{center}
    \includegraphics[scale=0.45]{../figures/wind_circulation.png}
  \end{center}
\end{frame}


\begin{frame}
  \frametitle{Tensão de cisalhamento do vento}
  \begin{center}
    \includegraphics[scale=0.38]{../figures/wind_stress.png}
  \end{center}
  \[
    (\tau_{wind_x}, \tau_{wind_y}) = \rho_{air} C_D U_{10}(u_a, v_a)
  \]
\end{frame}


\begin{frame}
  \frametitle{Relação entre tensão de cisalhamento do vento e forças de fricção.}
  \begin{block}{}
    Tensão de cisalhamento = -fluxo de momento
  \end{block}
  \begin{center}
    \includegraphics[scale=0.27]{../figures/friction_forces.png}
  \end{center}
\end{frame}


\begin{frame}
  \frametitle{Turbulência}
  \begin{center}
    \includegraphics[scale=0.31]{../figures/turbulence.png}
  \end{center}
\end{frame}


\subsection{Ekman}
\begin{frame}
  \frametitle{Balanço da força de Ekman}
  \begin{columns}
    \begin{column}{0.5\textwidth}
      \begin{center}
        \includegraphics[scale=0.25]{../figures/ekman_scheme.png}
      \end{center}
    \end{column}
    \begin{column}{0.5\textwidth}
      \[
        -fv_e = F_x = \frac{1}{\rho_{ref}}\pd{\tau_x}{z}
      \]
      \[
        fu_e = F_y = \frac{1}{\rho_{ref}}\pd{\tau_y}{z}
      \]
    \end{column}  \end{columns}
\end{frame}


\begin{frame}
  \frametitle{Transporte de Ekman}
  \begin{columns}
    \begin{column}{0.5\textwidth}
      \begin{center}
        \includegraphics[scale=0.25]{../figures/spiral_transport.png}
      \end{center}
    \end{column}
    \begin{column}{0.5\textwidth}
      \[
        \mathbf{M}_e = \int_{-D}^0 \rho_{ref}\mathbf{u}_edz
      \]
      \[
        \mathbf{M}_e = \frac{\tau_{wind} \times \hat{z}}{f}
      \]
    \end{column}  \end{columns}
\end{frame}


\begin{frame}
  \frametitle{Bombeamento (e sucção) de Ekman}
  Convergência e divergência no transporte de Ekman causa movimentos verticais.
  \begin{block}{}
    \[
      \pd{w}{z} = -\nabla_h . \mathbf{u}_e \text{, Assume }w=0 \text{ em } z=0
    \]
    \[
      w_{ek} = -\frac{1}{\rho_{ref}}\nabla_h . \mathbf{M}_e \text{, ou}
    \]
    \[
      w_{ek} = -\frac{1}{\rho_{ref}}\hat{z} . \nabla \times \left( \frac{\tau_{wind}}{f} \right)
    \]
  \end{block}
\end{frame}

\begin{frame}
  \frametitle{Distribuição do bombeamento de Ekman}
      \begin{center}
        \includegraphics[scale=0.42]{../figures/ekman_pump.png}
      \end{center}
\end{frame}


\subsection{Balanço Geostrófico}
\begin{frame}
  \frametitle{Balanço Geostrófico}
  {\scriptsize
  \begin{itemize}[<+-| alert@+>]
    \item Ocorre amplamente no interior do oceano e da atmosfera;
    \item É representado pelo equilíbrio entre a Força do Gradiente de Pressão
          e a Força de Coriolis;
    \item É um equilíbrio estacionário;
    \item A pressão em um ponto no interior do oceano é função do peso de água
          acima deste, que é função da densidade $\rho(S, T, P)$ e da altura da
          coluna d'água.
  \end{itemize}
  \pause
  \begin{block}{}
    \[
      u = -\frac{1}{f\rho}\pd{p}{y}
    \]

    \[
      v = -\frac{1}{f\rho}\pd{p}{x}
    \]
  \end{block}
}
\end{frame}


\begin{frame}
  \frametitle{Correntes}
  \begin{center}
    \includegraphics[scale=1.3]{../figures/currents_cartoon.png}
  \end{center}
\end{frame}

\subsection{Sumário}
\begin{frame}
\frametitle{Sumário}
  \begin{itemize}[<+-| alert@+>]
    \item A circulação gerada pelo vento (principalmente associada aos giros
          subtropicais), apresenta uma forte simetria entre o HN e HS.
    \item A circulação envolvendo os 1000 m do oceano é mais rápida, ocorre em
          escalas de bacias e medidas diretas podem ser usadas para a sua
          avaliação.
    \item A circulação termohalina é lenta, ocorre em escala global e é
          geralmente estudada através do conceito de massas de água.
  \end{itemize}
\end{frame}


\begin{frame}
  \frametitle{Desenhar o balanço de forcas!}
\end{frame}

\begin{frame}
  \frametitle{Dever de casa}
  \pause
    Chequem o portal!
\end{frame}

\end{document}
