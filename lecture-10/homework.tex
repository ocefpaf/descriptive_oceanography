\documentclass[12pt,portuguese,a4paper,pdftex]{article}
\usepackage{setspace}
\usepackage{lineno}
\usepackage[left=2cm,top=2cm,right=2cm]{geometry}
\usepackage[utf8]{inputenc}
\usepackage[T1]{fontenc}
\usepackage{lmodern}
\usepackage{epsf,epsfig}
\usepackage{url}
\usepackage[bookmarks=false,colorlinks=true,urlcolor={green},linkcolor={green},pdfstartview={XYZ null null 1.22}]{hyperref} % change colors for all references
\usepackage{amssymb,amsmath}
\usepackage{mathtools}
\usepackage{textcomp}
\everymath{\displaystyle}
\usepackage{float}

\newcommand{\pd}[2]{\frac{\partial #1}{\partial #2}} % partial derivatives

%% PDF metadata
\pdfinfo{% hyperref overrides this
  /Title    (Dever de Casa 10)
%   /Author   (Filipe Fernandes)
%   /Creator  (Filipe Fernandes)
%   /Producer (Filipe Fernandes)
  /Subject  (Oceanografia Física Descritiva)
  /Keywords (oceanografia)
}

\title{Dever de casa 10}
% \author{Filipe Fernandes}
\date{08-Nov-2013} % if comment gets today

\begin{document}
\maketitle
\doublespacing

A imagem do satélite NOAA abaixo foi obtida com sensor AVHRR, que mede
a radiação infra-vermelha e permite que a temperatura da superfície do oceano
seja mapeada remotamente.  Na imagem, os tons em vermelho denotam águas mais
quentes que os tons em amarelo.  A região de máximo gradiente está editada como
um linha grossa preta separando águas mais costeiras de águas mais oceânicas.

Os tons esverdeados/azuis nos entornos dos Cabo de São Tomé e Frio estão
associados a águas bastantes mais frias que tanto águas costeiras e oceânicas.
São a assinatura termal do fenômeno da ressurgência costeira.

\begin{figure}[H]
\vspace{1cm}
  \centerline{\includegraphics[scale=0.6]{../figures/satelite.png}}
\end{figure}

\begin{enumerate}
  \item Discorra e esquematize como o vento de Nordeste causa este fenômeno em
        Cabo Frio.  No esquema, represente o transporte de Ekman.  Apresente
        esquemas tanto no plano horizontal como uma seção transversal ao Cabo Frio.

  \item Explique como a superfície do oceano e as isopicnais respondem a este
        fenômeno.

  \item Discorra e esquematize como o vento de Sudoeste causa o fenômeno da
        subsidência em Cabo Frio.  No esquema, represente o transporte de
        Ekman.  Apresente esquemas tanto no plano horizontal como uma seção transversal ao Cabo Frio.

  \item Explique como a superfície do oceano e as isopicnais respondem a este
        fenômeno.

  \item Demonstre o balanço de forças no caso da ressurgência.
\end{enumerate}
\end{document}
