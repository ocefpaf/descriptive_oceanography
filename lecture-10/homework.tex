\documentclass[12pt,portuguese,a4paper,pdftex]{article}
\usepackage{setspace}
\usepackage{lineno}
\usepackage[left=2cm,top=2cm,right=2cm]{geometry}
\usepackage[utf8]{inputenc}
\usepackage[T1]{fontenc}
\usepackage{lmodern}
\usepackage{epsf,epsfig}
\usepackage{url}
\usepackage[bookmarks=false,colorlinks=true,urlcolor={green},linkcolor={green},pdfstartview={XYZ null null 1.22}]{hyperref} % change colors for all references
\usepackage{amssymb,amsmath}
\usepackage{mathtools}
\usepackage{textcomp}
\everymath{\displaystyle}
\usepackage{float}
\usepackage{subfigure}

\newcommand{\pd}[2]{\frac{\partial #1}{\partial #2}} % partial derivatives

%% PDF metadata
\pdfinfo{% hyperref overrides this
  /Title    (Dever de Casa 10)
  /Author   (Filipe Fernandes)
  /Creator  (Filipe Fernandes)
  /Producer (Filipe Fernandes)
  /Subject  (Oceanografia Física Descritiva)
  /Keywords (oceanografia)
}

\title{Dever de casa 10}
% \author{Filipe Fernandes}
\date{08-Nov-2013} % if comment gets today

\begin{document}
\maketitle
\doublespacing

A imagem do satélite NOAA abaixo foi obtida com sensor AVHRR, que mede
a radiação infra-vermelha e permite que a temperatura da superfície do oceano
seja mapeada remotamente.  Na imagem, os tons em vermelho denotam águas mais
quentes que os tons em amarelo.  A região de máximo gradiente está editada como
um linha grossa preta separando águas mais costeiras de águas mais oceânicas.

Os tons esverdeados/azuis nos entornos dos Cabo de São Tomé e Frio estão
associados a águas bastantes mais frias que tanto águas costeiras e oceânicas.
São a assinatura termal do fenômeno da ressurgência costeira.

\begin{figure}[H]
\vspace{1cm}
  \centerline{\includegraphics[scale=0.6]{../figures/satelite.png}}
\end{figure}

\begin{enumerate}
  \item Discorra e esquematize como o vento de Nordeste causa este fenômeno em
        Cabo Frio.  No esquema, represente o transporte de Ekman.  Apresente
        esquemas tanto no plano horizontal como uma seção transversal ao Cabo Frio.

  \item Explique como a superfície do oceano e as isopicnais respondem a este
        fenômeno.

  \item Discorra e esquematize como o vento de Sudoeste causa o fenômeno da
        subsidência em Cabo Frio.  No esquema, represente o transporte de
        Ekman.  Apresente esquemas tanto no plano horizontal como uma seção transversal ao Cabo Frio.

  \item Explique como a superfície do oceano e as isopicnais respondem a este
        fenômeno.

  \item Demonstre o balanço de forças no caso da ressurgência.
\end{enumerate}

\clearpage

\begin{enumerate}
  \item Ventos Nordeste são ventos propícios a Ressurgência devido ao transporte
        de Ekman resultante ser para a esquerda (no Hemisfério Sul - HS), levando
        água da superfície para longe da costa ``permitindo'' a entrada de água
        da termoclina em seu lugar.  Esse fenômeno só é possível pela conjunção
        do vento propício, batimetria com declinação suave e quando a
        Corrente do Brasil (CB) está afastada da Costa.

        Discorra e esquematize como o vento de Nordeste causa este fenômeno em
        Cabo Frio.  No esquema, represente o transporte de Ekman.  Apresente
        esquemas tanto no plano horizontal como uma seção transversal ao Cabo Frio.

  \item A superfície do oceano fica rebaixada próxima à costa e elevada ao
        largo.  Já as isopicnais ``afloram'' junto a costa (ressurgência) e
        rebaixam ao largo.

  \begin{figure}[H]
    \centering
    \mbox{
    \subfigure{\includegraphics[scale=0.5]{satelite-up.png}}\quad
    \subfigure{\includegraphics[scale=0.6]{upwelling.jpg}}}
  \end{figure}

  \clearpage

  \item Para ventos Sudeste o cenário é o contrário dos itens acima.

  \begin{figure}[H]
    \centering
    \mbox{
    \subfigure{\includegraphics[scale=0.5]{satelite-down.png}}\quad
    \subfigure{\includegraphics[scale=0.6]{downwelling.jpg}}}
  \end{figure}

  \item O balanço de forças no caso da ressurgência no HS é o Stress do Vento ao
        longo da Costa e Coriolis defletindo a água para à esquerda.

        \begin{align*}
          u_e &= \dfrac{1}{f\rho_{o}}\pd{\tau_y}{z} \\
          v_e &= -\dfrac{1}{f\rho_{o}}\pd{\tau_x}{z}
        \end{align*}


  \item[Extra:] Demonstre o balanço de forças para a Corrente Geostrófica.
        \begin{align*}
          u_g &= -\frac{1}{f\rho}\pd{p}{y} \\
          v_g &= -\frac{1}{f\rho}\pd{p}{x}
        \end{align*}

  \begin{figure}[H]
    \centering{\includegraphics[scale=0.8]{satelite-geos.png}}
  \end{figure}
\end{enumerate}

\end{document}
