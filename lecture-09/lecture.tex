% Title page.
\title[Aula 09]{Correntes oceânicas}
\subtitle{Coriolis, circulação gerada pelo vento e circulação termohalina}
\author[Filipe Fernandes]{Filipe P. A. Fernandes}
\institute[unimonte]{Centro Universitário Monte Serrat}
\date[Outubro 2013]{25 de Outubro 2013}

\logo{\includegraphics[scale=0.15]{../common/university_logo.png}}

\begin{document}

% The title page frame.
\begin{frame}[plain]
  \titlepage
\end{frame}

\section*{Outline}
\begin{frame}
\tableofcontents
\end{frame}

\section{Coriolis}
\begin{frame}
  \frametitle{Quem tem medo da rotação da Terra desce agora!}
  \begin{center}
    \includegraphics[scale=0.6]{./figures/calvin_rotation.png}
  \end{center}
\end{frame}


\begin{frame}
  \frametitle{Mas será que ela gira mesmo?}
  \begin{center}
    \includegraphics[scale=0.4]{./figures/galileo_goes_to_jail.jpg}
  \end{center}
\end{frame}


\begin{frame}
  \frametitle{Trilha estelar 01}
  \begin{center}
    \includegraphics[scale=0.45]{./figures/NCPtreeLosada.jpg}
%     http://apod.nasa.gov/apod/ap131023.html
  \end{center}
% North Celestial Pole at the center of all the star trail arcs recorded over a
% period of nearly 2 hours as a series of 30 second long, consecutive exposures
% on the night of October 5 near Almaden de la Plata, province of Seville, in
% southern Spain. The axis of rotation leads to the center of the concentric
% arcs in the night sky. For northern hemisphere the bright star Polaris is very
% close to the North Celestial Pole and so makes the short bright trail in the
% central gap between the leafy branches.
\end{frame}


\begin{frame}
  \frametitle{Trilha estelar 02}
  \begin{center}
    \includegraphics[scale=0.45]{./figures/2007_09_14-orion-lq_vangorp1200.jpg}
  \end{center}
% Made on September 14 from Montlaux, France, this wide-angle view nicely shows
% the stars near the celestial equator tracing nearly straight lines in
% projection, while stars north and south of the equator, respectively, appear
% to circle the north and south celestial poles. Featured are the stars of Orion
% (right of center), brilliant Venus rising (left) as bright star Sirius rises
% in the south (bottom center), and a polar orbiting Iridium satellite (upper
% left).  This picture was constructed from 477 consecutive 30 second digital
% exposures recorded over 4.3 hours.
\end{frame}


\begin{frame}
  \frametitle{Trilha estelar 03}
  \begin{center}
%     http://apod.nasa.gov/apod/ap040708.html
    \includegraphics[scale=4]{./figures/040523cruxa_seip_full.jpg}
% Gradually change the focus of the camera lens during the exposure, and you
% could end up with a dramatic picture like this one where the out-of-focus
% portion of the trail shows off the star's color.  Crux, the Southern Cross.
% Gacrux or gamma Crucis is the bright red giant star only 88 light-years
% distant that forms the top of the Cross seen here near top center. Acrux, the
% hot blue star at the bottom of the Cross is about 320 light-years distant.
% Actually a binary star system, Acrux is the alpha star of the compact Southern
% Cross and lies along a line pointing from Gacrux to the South Celestial Pole,
% off the lower right edge of the picture. Adding a separate short exposure to
% the end of the step-focussed trails to better show the positions of the stars
% themselves. The dark night skies above Namibia.
  \end{center}
\end{frame}


\begin{frame}
  \frametitle{Trilha estelar 04}
  \begin{center}
  \movie[showcontrols=true]{\centerline{\includegraphics[scale=0.7]{./figures/hawaiian_starlight_rotatingstarfield-apod.png}}}
  {./figures/hawaiian_starlight_rotatingstarfield-apod.flv}
  \end{center}
\end{frame}

\begin{frame}
  \frametitle{A Terra não gira apenas no próprio eixo}
  \begin{center}
    \includegraphics[scale=0.35]{./figures/image_0303_analemma0600ut_ayiomamitis_full.jpg}
  \end{center}
\end{frame}


\begin{frame}
  \frametitle{E o eixo não está onde gostaríamos...}
  \begin{center}
    \includegraphics[scale=0.7]{./figures/AxialTiltObliquity.png}
%     http://upload.wikimedia.org/wikipedia/commons/6/61/AxialTiltObliquity.png
  \end{center}
\end{frame}


\begin{frame}
  \frametitle{Formulando tudo isso}
  \begin{block}{Velocidades}
    \[
      \vec{u_{\mathbf{f}}} = \vec{u_{\mathbf{r}}} + \vec{\Omega} \times \vec{r}
    \]
  \end{block}

  \begin{block}{Acelerações}
    \[
      \vec{a_{\mathbf{f}}} = \vec{a_{\mathbf{r}}} + 2 \vec{\Omega} \times
      \vec{u_{\mathbf{r}}} - \Omega^2\vec{R}
    \]
  \end{block}
\end{frame}


\begin{frame}
  \frametitle{Só nos interessa a componente vertical}
  \begin{columns}
    \begin{column}{0.5\textwidth}
      \begin{itemize}
        \item $\Omega_x = 0$
        \item $\Omega_y = \Omega\cos\theta$
        \item $\Omega_z = \Omega\sin\theta$
      \end{itemize}
    \end{column}
    \begin{column}{0.5\textwidth}
      \begin{center}
        \includegraphics[scale=0.4]{./figures/coriolis_vertical.png}
      \end{center}
    \end{column}
  \end{columns}
\end{frame}

\begin{frame}
  \frametitle{De onde chegamos a $f = 2\Omega\sin{\theta}$?}
  \[
    2\vec{\Omega} \times \vec{u} =
    \begin{vmatrix}
    \vec{i} & \vec{j}           & \vec{k}\\
    0       & 2\Omega\cos\theta & 2\Omega\sin\theta\\
    u       & v                 & w
    \end{vmatrix}
  \]
  \pause
  \[
    2\vec{\Omega} \times \vec{u} =
    2\Omega [\xcancel{\vec{i}(w\cos\theta} - v\sin\theta) +
             \vec{j}u\sin\theta - \vec{k}u\cos\theta]
  \]
  \pause
  \begin{itemize}
    \item $(2\vec{\Omega} \times \vec{u})_x = -(2\Omega\sin\theta)v = -fv$
    \item $(2\vec{\Omega} \times \vec{u})_y = (2\Omega\sin\theta)u = fu$
    \item $(2\vec{\Omega} \times \vec{u})_z = \cancelto{*}{-(2\Omega\cos\theta)u}$
  \end{itemize}
\end{frame}


\begin{frame}
  \frametitle{23 horas, 56 minutos, 4.0916 segundos}
  \begin{itemize}[<+-| alert@+>]
    \item $\Omega = \dfrac{2\pi} {24 \times 60 \times 60} =
           7,272 \times 10^{-5}$ rad s$^{-1}$
    \item $\Omega^* = \dfrac{2\pi + 2\pi/360} {24 \times 60 \times 60} =
           7,292 \times 10^{-5}$ rad s$^{-1}$
  \end{itemize}
  \pause
  \small{* Esse é o $\Omega$ que devemos usar incluindo a rotação da Terra ao
         redor do Sol.  Existem outros efeitos menores como o Movimento do
         Sistema Solar na Galáxia e o Movimento de Galáxia no Universo.}
\end{frame}

\begin{frame}
  \frametitle{Alguns números}
  \begin{itemize}[<+-| alert@+>]
    \item $\Omega = 7,292 \times 10^{-5}$ rad s$^{-1}$
    \item Máxima aceleração de Coriolis (nos polos):

        $2\Omega \text{v} \sim 1,5 \times 10^{-4}\text{v}$

        v = 10000 m s$^{-1}  \rightarrow 1,5$ m s$^{-1}$
    \item $\dfrac{d\Omega}{dt} = 0$
  \end{itemize}
\end{frame}


\begin{frame}
  \frametitle{Onde foi parar $\Omega^2\vec{R}$?}

  \begin{columns}
    \begin{column}{0.5\textwidth}
    \small{
      Aceleração centrípeta no equador:
      $\Omega^2\text{R}_{\text{e}}^{*} = 0,0338$ m s$^{-2}$

      R$_{\text{equador}} -$ R$_{\text{polo}} \approx 21,4 \times 10^{3}$ m

      g$_{\text{equador}} -$ g$_{\text{polo}} \approx -0,052$ m s$^{-2}$
      }
    \end{column}
    \begin{column}{0.5\textwidth}
      \begin{center}
        \includegraphics[scale=0.6]{./figures/gravity_centrigugal_vectors.png}
      \end{center}
    \end{column}
  \end{columns}
\end{frame}

\section{Circulação circulação gerada pelo vento}
\begin{frame}
\frametitle{Correntes Oceânicas}
  \begin{itemize}[<+-| alert@+>]
    \item Uma corrente é caracterizada por um fluxo de água no oceano que
          apresenta uma distribuição coerente em termos de médias temporais;
    \item A importância de uma corrente é avaliada pelo seu transporte (tanto
          de volume como de calor) e pela variabilidade dos mesmos;
    \item As correntes oceânicas tem uma contribuição extremamente relevante no
          transporte de calor para os polos (principalmente em latitudes médias);
  \end{itemize}
\end{frame}


\begin{frame}
\frametitle{Correntes Oceânicas}
  \begin{itemize}[<+-| alert@+>]
    \item O estudo da circulação oceânica pode ocorrer através de observações
          in situ (navios, boias, instrumentos fundeados, satélites,
          derivadores), modelos analíticos e modelos numéricos.
  \end{itemize}
\end{frame}


\begin{frame}
\frametitle{Como são geradas}
  \begin{itemize}[<+-| alert@+>]
    \item As correntes oceânicas são geradas por dois mecanismos:
    \item {\bf Circulação gerada pela vento (0-1000 m):}
    \item Associada ao padrões de distribuição de ventos globais que formam os
          giros oceânicos em escalas de bacias;
    \item Processos desde variação sazonal até escalas climáticas (bacias).
   \end{itemize}
\end{frame}


\begin{frame}
\frametitle{Como são geradas}
  \begin{itemize}[<+-| alert@+>]
    \item {\bf Circulação termohalina (todo o oceano):}
    \item Processos relacionados as trocas de calor (aquecimento, resfriamento)
          e ou água doce (evaporação, precipitação).
    \item Processos em escalas climáticas (global).
   \end{itemize}
\end{frame}

\begin{frame}
  \frametitle{Circulação da superfície dos oceanos}
  \begin{center}
    \includegraphics[scale=0.5]{./figures/surface_circulation.png}
  \end{center}
\end{frame}

\begin{frame}
  \frametitle{Padrão de circulação de vento {\bf sem} rotação}
  \begin{center}
    \includegraphics[scale=0.55]{./figures/no_coriolis_winds.png}
  \end{center}
\end{frame}


\begin{frame}
  \frametitle{Padrão de circulação de vento {\bf com} rotação}
  \begin{center}
    \includegraphics[scale=0.4]{./figures/coriolis_winds.png}
  \end{center}
\end{frame}


\begin{frame}
  \frametitle{Ekman}
  \begin{center}
    \includegraphics[scale=0.6]{./figures/ekman_spiral.png}
  \end{center}
\end{frame}


\begin{frame}
  \frametitle{Circulação Geostrófica}
  \begin{center}
    \includegraphics[scale=0.6]{./figures/water_mound.png}
  \end{center}
\end{frame}


\begin{frame}
  \frametitle{Circulação Geostrófica}
  \begin{center}
    \includegraphics[scale=0.5]{./figures/geos_current.png}
  \end{center}
\end{frame}


\begin{frame}
  \frametitle{Intensificação de Correntes de Contorno Oeste}
  \begin{center}
    \includegraphics[scale=0.5]{./figures/western_intensification.png}
  \end{center}
\end{frame}


\begin{frame}
  \frametitle{Circulação da superfície dos oceanos}
  \begin{center}
    \includegraphics[scale=0.5]{./figures/surface_circulation.png}
  \end{center}
\end{frame}


\begin{frame}
  \frametitle{Dever de casa}
  \begin{center}
    \includegraphics[scale=0.55]{./figures/ocean_description.png}
  \end{center}

\scriptsize{

\href{http://oceanmotion.org/html/resources/coriolis.htm}{Circulação Inercial: \url{http://oceanmotion.org/html/resources/coriolis.htm}}

\href{http://www.britishpathe.com/video/gambling-with-gulf-stream-aka-gambling-on-the-gulf}{Corrente do Golfo: \url{http://www.britishpathe.com/video/gambling-with-gulf-stream-aka-gambling-on-the-gulf}}

\href{http://esminfo.prenhall.com/science/geoanimations/animations/26_NinoNina.html}{ENSO: \url{http://esminfo.prenhall.com/science/geoanimations/animations/26_NinoNina.html}}

}
\end{frame}

\end{document}
