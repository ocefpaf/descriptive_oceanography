% Title page.
\title[Aula 09]{Leis de conservação}
\subtitle{Correntes oceânicas}
\author[Filipe Fernandes]{Filipe P. A. Fernandes}
\institute[unimonte]{Centro Universitário Monte Serrat}
\date[Novembro 2012]{05 de Novembro 2012}

\logo{\includegraphics[scale=0.15]{../common/university_logo.png}}

\begin{document}

% The title page frame.
\begin{frame}[plain]
  \titlepage
\end{frame}

\section*{Outline}
\begin{frame}
\tableofcontents
\end{frame}

\section{Leis de conservação.}

\subsection{Conservação de massa}
\begin{frame}
\frametitle{Conservação de massa}
{\small
  \begin{itemize}[<+-| alert@+>]
    \item A lei de conservação da massa é utilizada em Oceanografia em diversas
          situações;
    \item Por exemplo, para um canal, esta lei exige que a massa de água que
          passa pela entrada menos a massa que passa pela saída do canal seja
          igual à variação da massa de água no interior do canal (normalmente
          indicada pela variação do nível da superfície da água).
    \item E no meio oceânico como um todo, a massa de água advinda de
          precipitações de chuva e neve mais a recebida por rios e derretimento
          de gelo é aproximadamente igual à massa de água perdida por
          evaporação mais a que congela.
  \end{itemize}
}
\end{frame}

\subsection{Ciclo hidrológico}

\subsection{Conservação do volume}
\begin{frame}
\frametitle{Conservação do volume}
  \begin{itemize}[<+-| alert@+>]
    \item Princípio da conservação do volume (ou equação da continuidade):
    \item Consequência do fato da compressibilidade da água ser pequena.
  \end{itemize}
  \begin{block}{}
  \[
    V_i + R + P = V_o + E
  \]
  \[
    Vo - V_i = R + P - E
  \]
  \end{block}
\end{frame}


\begin{frame}
\frametitle{Equação da continuidade}
\[
  \nabla_h . \mathbf{u} + \pd{w}{z} = 0
\]

ou,

\[
  \pd{u}{x} + \pd{v}{y} + \pd{w}{z} = 0
\]
\end{frame}


\subsection{Conservação do sal}
\begin{frame}
\frametitle{Conservação de salinidade}
\[
  \pd{S}{t} + (\mathbf{u} . \nabla_h)S + w\pd{S}{z} = \pd{}{z}\left( K_H\pd{S}{z} \right) + F_S
\]

ou (modelo de caixa),
\begin{block}{}
\[
  V_iS_i = V_oS_o
\]
\end{block}
\end{frame}


\subsection{Conservação de calor}
\begin{frame}
  \frametitle{Conservação de calor}
  \begin{itemize}[<+-| alert@+>]
  \item A temperatura das águas oceânicas varia no espaço e no tempo.
  \item Esta variação depende dos fluxos de calor que entram e saem em cada
        corpo d'água, e o cálculo desses fluxos se denomina "balanço térmico".
  \item Esses fluxos de radiação normalmente são denotados pela letra Q,
        com índices identificando cada componente do balanço térmico.
  \item Símbolo Q: fluxo: calor por unidade de tempo por unidade de área
  \end{itemize}
\end{frame}

\begin{frame}
\frametitle{Conservação de temperatura}
\[
  \pd{\theta}{t} + (\mathbf{u} . \nabla_h)\theta + w\pd{\theta}{z} = \pd{}{z}\left( K_H\pd{\theta}{z} \right) + F_{\theta}
\]
\end{frame}

\begin{frame}
  \frametitle{Conservação de calor}
  \begin{itemize}[<+-| alert@+>]
  \item $Q_S \rightarrow$ Radiação de onda curta (solar incidente);
  \item $Q_B \rightarrow$ Radiação de onda longa (emitida pelo oceano e
                          refletida por nuvens);
  \item $Q_H \rightarrow$ Calor por condução (Sensível);
  \item $Q_E \rightarrow$ Calor por evaporação/condensação (Latente);
  \item $Q_V \rightarrow$ Calor por advecção.
  \end{itemize}
\end{frame}

\begin{frame}
  \frametitle{Conservação de calor}
  \begin{block}{}
    \[
      Q_{net} = Q_S + Q_V + Q_B + Q_H + Q_E + Q_T
    \]
  \end{block}
\end{frame}

\section{Teoria da radiação eletromagnética}
\begin{frame}
  \frametitle{Teoria da radiação eletromagnética}
  Lei de Stefan diz que um corpo com temperatura $T$ emite radiação  segundo:
  \begin{block}{}
    \[
      Q \sim \sigma T^4
    \]
  \end{block}
  onde, $\sigma \approx 5.7338$ J m$^{-2}$\textdegree{K}$^{-4}$s$^{-1}$
\end{frame}

\subsection{Radiação de ondas curtas}
\begin{frame}
  \frametitle{Ondas curtas}
  \begin{itemize}[<+-| alert@+>]
    \item 48\% atinge o mar, dos quais
    \item 29\% é radiação direta (do sol)
    \item 19\% é radiação indireta (após espalhamento na atmosfera)
  \end{itemize}
\end{frame}

\begin{frame}
  \frametitle{Fatore que afetam $Q_S$}
  \begin{itemize}[<+-| alert@+>]
    \item Duração do dia (depende da estação do ano e da latitude).
    \item Elevação do sol (Densidade de energia é proporcional ao seno do
          ângulo de elevação do sol).
    \item Absorção na atmosfera (devido a moléculas de gás, poeira , vapor
          d'água, ...)
    \item Efeito de nuvens: que diminui a radiação direta e daí a total,
          embora aumente a radiação indireta.
    \item Efeito de ondas na superfície do mar. (que modificam os ângulos de
          incidência dos raios de sol).
  \end{itemize}
\end{frame}

\subsection{Radiação de ondas longas}
\begin{frame}
  \frametitle{Ondas longas}
  \begin{block}{}
    $Q_B$ proporcional à temperatura absoluta da superfície elevado à
          quarta potência.
  \end{block}
\end{frame}

\begin{frame}
  \frametitle{Fatores que influenciam $Q_B$}
  {\footnotesize
  \begin{itemize}[<+-| alert@+>]
    \item Espessura das nuvens. Quanto maior a espessura da nuvem, menos é o
          calor que escapa para o espaço.
    \item A altura das nuvens, o que determina a temperatura em que a nuvem
          irradia calor de volta ao oceano. Nuvens altas são mais frias do que
          as nuvens baixas.
    \item Quantidade de vapor atmosférico (quanto mais úmido, menor a quantidade
          de calor que escapa para o espaço).
    \item Temperatura da água (quanto maior a temperatura maior a radiação).
    \item Cobertura de gelo e neve. O gelo emite como um corpo negro, mas esfria
          muito mais rápido do que as áreas de oceano aberto.  Mares cobertos de
          gelo são isolados da atmosfera.
  \end{itemize}
}
\end{frame}

\subsection{Evaporação}
\begin{frame}
  \frametitle{Fatores que influenciam $Q_E$}
  \begin{itemize}
    \item Vento (aumento do vento aumenta a evaporação).
    \item Umidade relativa do ar (ar seco aumenta a evaporação).
  \end{itemize}
\end{frame}

\begin{frame}
  \frametitle{Evaporação}
  \begin{block}{}
    Evaporação é muito importante, mas difícil de determinar diretamente.
    Implica em perda de volume d'água e de calor.
  \end{block}
\end{frame}

\subsection{Condução de calor}
\begin{frame}
  \frametitle{Condução}
  \begin{block}{}
    Altamente dependente da diferença de temperatura entre o oceano e a atmosfera.
  \end{block}

\end{frame}

\begin{frame}
  \frametitle{$Q_{net}$}
  \begin{center}
    \includegraphics[scale=0.38]{../figures/convex_narr_fluxes.png}
  \end{center}
\end{frame}


\begin{frame}
  \frametitle{Distribuição meridional}
  \begin{center}
    \includegraphics[scale=0.48]{../figures/Qnet.png}
  \end{center}
\end{frame}


\subsection{Razão de Bowen}
\begin{frame}
  \frametitle{Razão de Bowen}
  \begin{block}{}
    \[
      R = \frac{Q_H}{Q_E}
    \]

    \[
      R = 0.062 \frac{(t_s - t_a)}{e_s - e_a}
    \]
  \end{block}
\end{frame}

\section{Correntes Oceânicas}
\begin{frame}
\frametitle{Correntes Oceânicas}
  \begin{itemize}[<+-| alert@+>]
    \item Uma corrente é caracterizada por um fluxo de água no oceano que
          apresenta uma distribuição coerente em termos de médias temporais;
    \item A importância de uma corrente é avaliada pelo seu transporte (tanto
          de volume como de calor) e pela variabilidade dos mesmos;
    \item As correntes oceânicas tem uma contribuição extremamente relevante no
          transporte de calor para os polos (principalmente em latitudes médias);
  \end{itemize}
\end{frame}


\begin{frame}
\frametitle{Correntes Oceânicas}
  \begin{itemize}[<+-| alert@+>]
    \item O estudo da circulação oceânica pode ocorrer através de observações
          in situ (navios, boias, instrumentos fundeados, satélites,
          derivadores), modelos analíticos e modelos numéricos.
  \end{itemize}
\end{frame}


\begin{frame}
\frametitle{Como são geradas}
  \begin{itemize}[<+-| alert@+>]
    \item As correntes oceânicas são geradas por dois mecanismos:
    \item {\bf Circulação gerada pela vento (0-1000 m):}
    \item Associada ao padrões de distribuição de ventos globais que formam os
          giros oceânicos em escalas de bacias;
    \item Processos desde variação sazonal até escalas climáticas;
    \item Escala de bacias.
   \end{itemize}
\end{frame}


\begin{frame}
\frametitle{Como são geradas}
  \begin{itemize}[<+-| alert@+>]
    \item {\bf Circulação termo-halina (todo o oceano):}
    \item Processos relacionados as trocas de calor (aquecimento, resfriamento)
          e ou água doce (evaporação, precipitação).
    \item Processos em escalas climáticas;
    \item Escala global.
   \end{itemize}
\end{frame}



% \begin{frame}
%   \frametitle{Dever de casa}
%   \pause
%   \begin{block}{}
%     Estudar para a prova!
%   \end{block}
% \end{frame}

\end{document}
