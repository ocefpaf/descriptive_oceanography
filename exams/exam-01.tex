% Style.
\documentclass[letterpaper,portuguese,12pt,pdftex]{exam}

\usepackage{setspace}
\usepackage{lineno}
\usepackage[left=2.5cm,top=3cm,right=2.5cm]{geometry}

% Portuguese.
\usepackage[brazil]{babel}
\usepackage[T1]{fontenc}
\usepackage[utf8x]{inputenc}
\usepackage{textcomp}

% Font.
\usepackage{lmodern}

% Figures.
\usepackage{epsf,epsfig}

% Bibtex and extras.
\usepackage{natbib}
\usepackage{url}
\usepackage[bookmarks=false,colorlinks=true,urlcolor={green},linkcolor={green},pdfstartview={XYZ null null 1.22}]{hyperref}

% Math.
\usepackage{amssymb,amsmath}
\usepackage{mathtools}
\everymath{\displaystyle}

% Exam.
\addpoints
\printanswers
% \noprintanswers
\usepackage{color}
\definecolor{SolutionColor}{rgb}{0.8,0.9,1}
\shadedsolutions
\renewcommand{\solutiontitle}{\noindent\textbf{Solução:}\par\noindent}
\pagestyle{headandfoot}
\footer{}{Página \thepage\ de \numpages}{}
\boxedpoints
\pointsinrightmargin
\pointpoints{ponto}{pontos}
\hqword{Questão}
\hpword{Pontos}
\hsword{Nota}
% \qformat{\textbf{Question\thequestion}\quad(\thepoints)\hfill}

% User commands.
\newcommand{\pd}[2]{\frac{\partial #1}{\partial #2}}

% PDF metadata.
\pdfinfo{% hyperref overrides this
  /Title    (Prova 01 -- Oceanografia Física Descritiva)
  /Author   (Filipe Fernandes)
  /Creator  (Filipe Fernandes)
  /Producer (Filipe Fernandes)
  /Subject  (prova)
  /Keywords (prova, oceanografia)
}

% Front page.
\title{Prova 01 -- Oceanografia Física Descritiva}
\author{Prof. Filipe Fernandes}
\date{04-Out-2013}

\begin{document}
\maketitle
\doublespacing

\vspace{1cm}
\hbox to \textwidth{Nome e número de matrícula:\enspace\hrulefill}
\vspace{1cm}

\begin{minipage}{.8\textwidth}
Esse exame incluí \numquestions\ questões. O número total de pontos é
\numpoints.  % 25

A interpretação das questões faz parte da prova.  Qualquer problema com a
redação das mesmas será resolvido {\bf após} a prova e não durante.

\end{minipage}
\vspace{0.5cm}

\begin{questions}

  \question
  \begin{parts}
  \part[2]
  Explique como a {\bf densidade} varia com a {\bf pressão}, {\bf temperatura}
  e {\bf salinidade}.

  \begin{solution}
    \begin{itemize}
      \item Pressão: Aumenta quase linearmente com a pressão.
    \end{itemize}
  \end{solution}

  \part[2]
  O que são e como se relacionam: {\bf Densidade ($\rho$)}, {\bf sigma ($\sigma$)},
  {\bf sigma-t ($\sigma_t$)} e  {\bf sigma-theta ($\sigma_{\theta}$)}.

  \begin{solution}
    \begin{itemize}
      \item $\rho = \rho(S, t, p)$
      \item $\sigma = \rho - 1000$
      \item $\sigma_t = \sigma_t(S, t, p=0)$
      \item $\sigma_{\theta} = \sigma_{\theta}(S, \theta, p=0)$
    \end{itemize}
  \end{solution}

  \end{parts}

  \question

  \begin{parts}
  \part[4]
  Explique porque a {\bf temperatura superficial} nas proximidades do
  {\bf equador} é ligeiramente mais {\bf fria} que as águas ao seu redor.
  O mesmo acontece com a {\bf salinidade}? Justifique a sua resposta!

  \begin{solution}
    Modelo de ressurgência global para t.
    Chuvas para S.
  \end{solution}

  \part[4]
  O que são propriedades {\bf conservativas} e {\bf não-conservativas}?
  De {\bf 2} exemplos de cada uma e discorra como podemos usar essas
  propriedades ao nosso favor para {\bf traçar massas d'água}.

  (Ponto extra: Temperatura {\it in-situ} é verdadeiramente conservativa?)

  \begin{solution}
    TODO
  \end{solution}

  \part[4]
  Cite 1 exemplo para cada escala de fenômenos: {\bf pequena},
  {\bf meso} e {\bf grande-escala} que ocorrem nos oceanos.  Na sua resposta
  comente sobre como {\bf observar/estudar} cada um dos fenômenos que você
  escolheu.

  (Exemplo: Estudo de {\bf ressurgência costeira}, fenômeno de {\bf meso-escala},
  usando seções de {\bf XBTs quinzenais} -- que é a frequência média de passagem
  de frentes na região Sul/Sudeste da costa Brasileira.)

  \end{parts}

  \question
  \begin{parts}
  \part[3]
  Esquematize {\bf perfis verticais} oceânicos típicos de {\bf temperatura},
  {\bf salinidade} e {\bf densidade} para regiões sub-tropicais.  Nomeie as
  {\bf 3 camadas} de estratificação em seu perfil e teça comentários sobre a sua
  {\bf estabilidade}.

  \begin{solution}
    O perfil de densidade é quase um espelho da temperatura porque, apesar da
    densidade variar mais com a salinidade, a variação da temperatura é muito
    maior.
  \end{solution}

  \part[2]
  Explique o que é a {\bf termoclina sazonal}.  Explique a sua {\bf formação} e
  como essa se diferencia da {\bf termoclina permanente}.

  \begin{solution}
  \end{solution}

  \end{parts}


  \question[4]
  Comente sobre a {\bf veracidade} de cada afirmação abaixo:
  \begin{itemize}
    \item[a)] Em regiões subtropicais nota-se uma tendência para que as
              temperaturas superficiais da água do mar sejam maiores próximo à
              costa oeste do que próximo à costa leste dos oceanos.
    \item[b)] A temperatura potencial da água do mar é, em geral, maior ou igual
              à temperatura {\it in-situ}.
    \item[c)] A termoclina sazonal é, em geral, mais profunda durante o verão e
              mais rasa durante o inverno.
    \item[d)] A termoclina sazonal é um fenômeno típico da região equatorial dos
              oceanos.
  \end{itemize}

\end{questions}

\end{document}
