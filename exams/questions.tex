\begin{questions}
  A maioria dos líquidos atingem uma densidade máxima no ponto de congelamento,
  mas a água pura é uma exceção.  Em qual temperatura a água pura atinge uma
  máxima densidade?  Esta temperatura se aplica para a água do mar?
\end{questions}

\begin{questions}
  Quais são as propriedades da água que, provavelmente, evitam que a superfície
  da Terra alcance extremos de temperatura?

  \begin{parts}
    \part
    Partindo da constância da composição para os maiores constituintes da água
    do mar, estes se comportam como conservativos ou não conservativos? Justificar.

    \part
    A clorinidade é uma propriedade conservativa? Comentar sua resposta.
  \end{parts}
\end{questions}

\begin{questions}
  Variações diurnas diárias da temperatura sobre a terra são algumas vezes
  mensuráveis em dezenas de graus, mas nos oceanos estas quantidades não são
  mais que uns poucos graus, exceto em águas muito rasas.  Sugerir três principais
  razões para isto.
\end{questions}

\begin{questions}
  A ampla estrutura termal dos oceanos nos conduz a reconhecer três camadas
  principais.  Definir estas camadas e resumir suas características.

  \begin{solution}
    Em termos gerais, como mostrado na Figura 2.9 (The Open University, 1993),
    pode-se reconhecer primeiro a camada de mistura, na qual as termoclinas diurna e
    sazonal ocorrem em certas latitudes e períodos do ano.  Abaixo desta repousa a
    termoclina permanente ou principal, onde a temperatura decai relativamente
    rápido desde mais que 15\textcelsius{} para aproximadamente 5\textcelsius{}.
    Finalmente, o volume do oceano profundo, a camada profunda desde ao redor de
    1000 metros até o fundo, é caracterizado por temperaturas  menores que
    aproximadamente 3\textcelsius{}.  Esta estrutura de três camadas não é observada
    em altas latitudes, onde os perfis de temperatura estão praticamente na vertical
    (Fig. 2.7c – The Open University, 1993).
  \end{solution}
\end{questions}

\begin{questions}
  Baseado em seus conhecimentos,  o que é uma propriedade conservativa.
  Considerando este conceito, pode-se dizer que a temperatura potencial é uma
  propriedade conservativa comparada com a temperatura ``{\it in-situ}''?
  Justificar.

  \begin{solution}
    Há duas razões principais para  o diagrama T-S ser uma ferramenta eficiente para
    identificar e traçar  massas de água.  Primeiro, temperatura e salinidade são
    facilmente medidas.  Segundo, assim que a água está fora de contato com a
    atmosfera, isto é uma vez que ela deixou a camada de mistura na superfície e
    está no corpo principal do oceano, estas propriedades podem somente ser
    alteradas por mistura com águas com características de T e S diferentes.

    Estritamente falando, a resposta é sim.  Temperatura ``{\it in-situ}'' pode ser
    alterada por outros processos do que por mistura, especialmente por compressão adiabática ou expansão.  Temperatura potencial tem sido corrigida deste efeito,
    tal que esta é uma propriedade conservativa verdadeira.  Por esta razão, o uso
    de diagramas T-S está sendo amplamente substituído pelo uso de diagramas
    $\theta$-S.
  \end{solution}
\end{questions}

\begin{questions}
  \begin{parts}
    \part
    Explicar porquê o perfil de temperatura de um lago de água doce não poderá
    apresentar temperaturas decrescendo com a profundidade para valores menores
    que 4\textcelsius.

    \part
    Explicar porquê a água mais fria pode sobrepor a água em 4\textcelsius em um
    lago de água doce em uma situação gravitacionalmente estável.  Poderia tal
    situação se desenvolver nos oceanos?
  \end{parts}
\end{questions}

\begin{questions}
  Assinale a única alternativa correta e justificar.
  \begin{itemize}
    \item[a)] Os rios suprem os oceanos com cerca de 90\% do volume de água
              perdido por esses últimos através da evaporação.
    \item[b)] A camada de haloclina é uma característica do perfil vertical de
              salinidade que ocorre permanentemente em regiões polares.
    \item[c)] Na escala prática a salinidade é definida em termos da razão de
              condutividades elétricas da amostra de água do mar e de uma
              solução de cloreto de potássio.
    \item[d)] A salinidade na camada de superfície nos oceanos depende da
              evaporação, sendo por essa razão maior nas regiões equatoriais.
  \end{itemize}
\end{questions}

\begin{questions}
  Considere as Seções verticais sul-norte de temperatura potencial, salinidade,
  $\sigma_{\theta}$ e oxigênio dissolvido no Oceano Atlântico Oeste.

  \begin{parts}
    \part
    Tecer comentários sobre as referidas seções verticais.

    \part
    Esboçar perfis verticais de salinidade e densidade em baixas, médias e altas latitudes, comentando sobre suas características.
  \end{parts}
\end{questions}

\begin{questions}
  Considere a Equação Internacional de Estado (EOS-80), genericamente dada por:
  $\rho_{(S, T, p)} = \rho_{(S, T, 0)}(1 - p / K_{S,T, p})$

  \begin{parts}
    \part
    Discutir sobre o significado físico desta Equação e comentar, sucintamente,
    como a   mesma foi obtida.

    \part
    A partir da EIE-80 determinar a expressão para o cálculo do coeficiente de
    expansão térmica ($\alpha$).
  \end{parts}
\end{questions}

\begin{questions}
  Partindo da forma implícita da equação de estado da água do mar

  \[
    \rho  =  \rho(S, T, p)
  \]

  obter uma equação de estado na forma diferencial, em função dos coeficientes
  de expansão térmica ($\alpha$), contração salina ($\beta$) e compressibilidade
  bárica ($\kappa$).  Interpretar fisicamente a equação resultante.
\end{questions}

\begin{questions}
  \begin{parts}
    \part
    Baseado em seus conhecimentos, quais são as vantagens em se usar as formas
    simplificadas da equação de estado da água do mar?

    \part
    Como pesquisador, em quais situações ou regiões de estudo você usaria as
    formas simplificadas da equação de estado da água do mar.  Exemplificar.
  \end{parts}
\end{questions}



\question[5]
Explique o que são {\bf massas d'água}.  Como são {\bf formadas} e como podemos
{\bf rastreá-las} nos oceanos.  Faça uso da equação abaixo em sua explicação.
Qual termo {\bf não} pode existir para que a propriedade em questão seja
{\bf conservativa}.

\[
  \underbrace{\pd{S}{t}}_a = \underbrace{\vec{u}\nabla S}_b + \underbrace{\phi}_c
\]

\begin{solution}
 Massas d'água é uma porção de água do mar com uma origem determinada.
 Formam-se na superfície do mar devido às forças atmosféricas ganhando o seu $T$
 e $S$ característico pelo qual podemos rastreá-las.  Na equação temos que
 cortar o termo que corresponde à fonte ou sorvedouro $\phi$.
\end{solution}


\question[3]
Os digramas TS abaixo (Fig. \ref{fig:mistura}) representam dois tipos de mistura
({\bf Diapicnal e isopicnal}). {\bf Assinale} qual diagrama representa cada tipo
de mistura.  {\bf Explique} a diferença entre esse dois tipos e {\bf comente}
qual deles necessita de energia para exercer {\bf trabalho vertical}.

\begin{figure}[ht]
  \centering
  \includegraphics[scale=0.65]{./mixing_types.png}
  \caption{Diagrama TS.}
  \label{fig:mistura}
\end{figure}

\begin{solution}
 A) Isopicnal: Mistura por difusão ao longo da superfície de mesma densidade.\\
 B) Diapicnal: Mistura mecânica (com trabalho) ao longo de superfícies de densidade.\\
\end{solution}

\clearpage

\question
A figura \ref{fig:giro} é uma representação esquemática de um típico {\bf Giro
Subtropical}, como os que formam as correntes do Brasil e corrente do Golfo.


\begin{figure}[ht]
  \centering
  \includegraphics{./agua_central.png}
  \caption{Giro Subtropical.}
  \label{fig:giro}
\end{figure}

\begin{parts}
  \part[3]
  Explique como se dá o ``{\bf transporte de Ekman}'' assinalado pelas setas
  transparentes.  Sua resposta deve conter o {\bf balanço de forças},
  {\bf espiral de Ekman}, e {\bf transporte} integrado na coluna d'água.

  \begin{solution}
   O transporte de Ekman se da pelo balanço entre as forças de atrito e Coriolis.
   O transporte integrado na coluna d'água é o somatório da espiral na profundidade
   da influência de Ekman.
  \end{solution}

  \part[1]
  Usando conceitos de {\bf convergência} e {\bf divergência}, explique como o
  transporte de  Ekman gera a {\bf alta pressão} do Giro subtropical.  Qual é o
  papel desse fenômeno na formação da {\bf água central}?

  (Dica: pense em bombeamento de Ekman.)

  \begin{solution}
   Devido à circulação vento temos sempre transporte de Ekman para o centro do Giro.
   Assim, temos água ``acumulando'' no centro gerando a alta pressão.  Para compensar
   essa água, temos o bombeamento de Ekman para o interior do oceano ``empurrando''
   a Água central.
  \end{solution}

\end{parts}

\question
{\bf Modelo em caixa de um estuário (Figura \ref{fig:box}).}

Um estuário tem os transportes de {\bf entrada} e {\bf saída} $T_i$ e $T_o$
respectivamente.  Durante um ano normal, o transporte oriundo dos {\bf rios}
nesse estuário é de 50 m$^3$ s$^{-1}$.  A {\bf evaporação} total, durante esse
ano típico, excede a {\bf precipitação} em 4 m$^3$ s$^{-1}$.  As {\bf salinidades}
estuarina e oceânica são de 30 e 36 g kg$^{-1}$ respectivamente.

Usando a equação da {\bf conservação de volume} (equação \ref{eq:conservation0} ou \ref{eq:conservation1}) e a equação de {\bf conservação de sal} \ref{eq:salt}
responda.

\begin{align}
  \underbrace{T_i + R + P}_{\text{Volume de entrada}} &= \underbrace{T_o + E}_{\text{Volume de saída}} \text{, ou} \label{eq:conservation0}\\
  T_i + R &= T_o + (E-P) \label{eq:conservation1}\\
  T_iS_i &= T_oS_o \label{eq:salt}
\end{align}

\begin{figure}[ht]
  \centering
  \includegraphics[scale=0.5]{./box_model.png}
  \caption{Modelo em caixa de um estuário.}
  \label{fig:box}
\end{figure}

\begin{parts}
  \part[2]
  Calcule o transporte de entrada $T_i$.

  \begin{solution}
   \[
    T_i = \frac{S_o}{S_i - S_o}\left[R-(E-P)\right]
   \]
   \[
    T_i = \frac{30}{6}[50-4] = 230 \text{ m}^3 \text{s}^{-1}
   \]
  \end{solution}


  \part[2]
  Durante uma {\bf seca} o transporte do rio caí para 25 m$^3$ s$^{-1}$, e a
  evaporação {\bf excede} a precipitação em 6 m$^3$ s$^{-1}$.  Assumindo que a
  quantidade de sal que entra no estuário não muda ($S_i$), calcule a {\bf nova
  salinidade} dentro do estuário ($S_o$).

  \begin{solution}
   \begin{align*}
    S_o &= \frac{T_iS_i}{T_o}\\
    T_o &= T_i + R - (E-P)
   \end{align*}
   \[
      T_o = 230 + 25 - 6 = 249 \text{ m}^3 \text{s}^{-1}
   \]
   \[
      S_o = \frac{230 \times 36}{249} = 33.25 \text{ g kg}^{-1}
   \]
  \end{solution}

  \part[4]
  Se assumirmos que o volume do estuário é {\bf constante} e definido Tempo de
  de Residência ($RT$) como:
  \[
    RT = \frac{\text{Volume total}}{\text{Volume de entrada}}
  \]

  Qual é a {\bf taxa} do tempo de residência antes da seca ($RT_{\text{antes}}$)
  e depois da seca ($RT_{\text{depois}}$)?

  \begin{solution}
    \[
      \text{Volume de entrada} = T_i + R
    \]
   \[
      \frac{RT_1}{RT_2} = \frac{\text{Entrada}_2}{\text{Entrada}_1} = \frac{230+25}{230+50} = 0.91
   \]
  \end{solution}

\end{parts}

\question
A corrente do Brasil nas proximidades de 20\textdegree{S} tem velocidades
superficiais na ordem de 0.65 m s$^{-1}$ e uma largura média de $\sim$ 100 km.

Sabendo que a velocidade {\bf geostrófica} é dada por:
 \[
    fv = g\tan(\theta),
 \]


onde:
\begin{itemize}
 \item $v$ é a velocidade meridional;
 \item $f$ o parâmetro de Coriolis dado por $2\Omega\sin(\text{lat})$ e,
  \begin{enumerate}
   \item[] $\Omega = 7.29 \times 10^{-5}$ rad s$^{-1}$;
   \item[] lat $\equiv$ latitude local;
  \end{enumerate}

 \item $g$ a aceleração da gravidade;
 \item $\theta$ o ângulo formado pela declinação da superfície livre.
\end{itemize}

\begin{figure}[ht]
  \centering
  \includegraphics[scale=0.7]{./BC_slope.png}
  \caption{Desenho esquemático da declinação gerada pela alta pressão do giro
           Subtropical.}
  \label{fig:BC}
\end{figure}

\begin{parts}

\part[3]
Estime a magnitude e direção da declinação da superfície livre que cruza a
corrente do Brasil.

\clearpage

\begin{solution}

\begin{align*}
  f &= 2\Omega\sin(-20)\\
  f &= 0.0001458423 \sin(-0.349)\\
  f &= -4.988\times 10^{-5} \text { s}^{-1}\\
  &\downarrow\\
 \tan\phi &= \frac{fv}{g}\\
 \tan\phi &= \frac{-4.988\times 10^{-5} \text { s}^{-1} \times 0.65 \text{ m s}^{-1}}{9.8 \text{ m s}^{-2}}\\
  \phi &= \arctan{-3.642 \times ^{-6}}\\
  \phi &= -0.000208^{\circ}\\
  \text{ ou,}\\
 \Delta h &= 10^5\text{ m} \times -3.642 \times ^{-6}\\
 \Delta h &= 0.36 \text{ m}
\end{align*}

\end{solution}

\end{parts}

\question
Sabemos que as componentes individuais de {\bf radiação térmica} são:
  \begin{itemize}
  \item $Q_S \rightarrow$ Radiação de onda curta (solar incidente);
  \item $Q_B \rightarrow$ Radiação de onda longa (emitida pelo oceano e
                          refletida por nuvens);
  \item $Q_H \rightarrow$ Calor por condução (Sensível);
  \item $Q_E \rightarrow$ Calor por evaporação/condensação (Latente);
  \item $Q_V \rightarrow$ Calor por advecção.
  \end{itemize}

E que o calor {\bf total} devido a radiação {\bf atmosférica} em uma certa
parcela de água se dá por:
    \[
      Q_{net} = Q_S + Q_B + Q_H + Q_E
    \]

\begin{parts}
 \part[1]
 Estimativas de satélite de temperatura superficial do mar (TSM) observadas por
 um certo período de tempo mostram que, uma camada à 25 m de profundidade,
 chamada de ``camada de mistura'' (localizada acima da termoclina) está
 {\bf esquentando mais que devia} quando computamos o balanço $Q_{net}$ vertical
 (ou atmosférico) {\bf chegando}.

  Usando o que você sabe sobre balanço de calor re-escreva a equação de $Q_{net}$
  incluindo o termo responsável por esse aquecimento (explique o porque).

  \begin{solution}
   Apenas o transporte de calor advectado pode ser o responsável por esse
   aquecimento anômalo $ Q_V$
  \end{solution}

  \part[1]
  Esquematize um diagrama com o que pode estar ocorrendo.

  \begin{solution}
   Inserir figura.
  \end{solution}

\end{parts}
