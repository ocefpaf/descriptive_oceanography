\begin{questions}
  A maioria dos líquidos atingem uma densidade máxima no ponto de congelamento,
  mas a água pura é uma exceção.  Em qual temperatura a água pura atinge uma
  máxima densidade?  Esta temperatura se aplica para a água do mar?
\end{questions}

\begin{questions}
  Quais são as propriedades da água que, provavelmente, evitam que a superfície
  da Terra alcance extremos de temperatura?

  \begin{parts}
    \part
    Partindo da constância da composição para os maiores constituintes da água
    do mar, estes se comportam como conservativos ou não conservativos? Justificar.

    \part
    A clorinidade é uma propriedade conservativa? Comentar sua resposta.
  \end{parts}
\end{questions}

\begin{questions}
  Variações diurnas diárias da temperatura sobre a terra são algumas vezes
  mensuráveis em dezenas de graus, mas nos oceanos estas quantidades não são
  mais que uns poucos graus, exceto em águas muito rasas.  Sugerir três principais
  razões para isto.
\end{questions}

\begin{questions}
  A ampla estrutura termal dos oceanos nos conduz a reconhecer três camadas
  principais.  Definir estas camadas e resumir suas características.

  \begin{solution}
    Em termos gerais, como mostrado na Figura 2.9 (The Open University, 1993),
    pode-se reconhecer primeiro a camada de mistura, na qual as termoclinas diurna e
    sazonal ocorrem em certas latitudes e períodos do ano.  Abaixo desta repousa a
    termoclina permanente ou principal, onde a temperatura decai relativamente
    rápido desde mais que 15\textcelsius{} para aproximadamente 5\textcelsius{}.
    Finalmente, o volume do oceano profundo, a camada profunda desde ao redor de
    1000 metros até o fundo, é caracterizado por temperaturas  menores que
    aproximadamente 3\textcelsius{}.  Esta estrutura de três camadas não é observada
    em altas latitudes, onde os perfis de temperatura estão praticamente na vertical
    (Fig. 2.7c – The Open University, 1993).
  \end{solution}
\end{questions}

\begin{questions}
  Baseado em seus conhecimentos,  o que é uma propriedade conservativa.
  Considerando este conceito, pode-se dizer que a temperatura potencial é uma
  propriedade conservativa comparada com a temperatura ``{\it in-situ}''?
  Justificar.

  \begin{solution}
    Há duas razões principais para  o diagrama T-S ser uma ferramenta eficiente para
    identificar e traçar  massas de água.  Primeiro, temperatura e salinidade são
    facilmente medidas.  Segundo, assim que a água está fora de contato com a
    atmosfera, isto é uma vez que ela deixou a camada de mistura na superfície e
    está no corpo principal do oceano, estas propriedades podem somente ser
    alteradas por mistura com águas com características de T e S diferentes.

    Estritamente falando, a resposta é sim.  Temperatura ``{\it in-situ}'' pode ser
    alterada por outros processos do que por mistura, especialmente por compressão adiabática ou expansão.  Temperatura potencial tem sido corrigida deste efeito,
    tal que esta é uma propriedade conservativa verdadeira.  Por esta razão, o uso
    de diagramas T-S está sendo amplamente substituído pelo uso de diagramas
    $\theta$-S.
  \end{solution}
\end{questions}

\begin{questions}
  \begin{parts}
    \part
    Explicar porquê o perfil de temperatura de um lago de água doce não poderá
    apresentar temperaturas decrescendo com a profundidade para valores menores
    que 4\textcelsius.

    \part
    Explicar porquê a água mais fria pode sobrepor a água em 4\textcelsius em um
    lago de água doce em uma situação gravitacionalmente estável.  Poderia tal
    situação se desenvolver nos oceanos?
  \end{parts}
\end{questions}

\begin{questions}
  Assinale a única alternativa correta e justificar.
  \begin{itemize}
    \item[a)] Os rios suprem os oceanos com cerca de 90\% do volume de água
              perdido por esses últimos através da evaporação.
    \item[b)] A camada de haloclina é uma característica do perfil vertical de
              salinidade que ocorre permanentemente em regiões polares.
    \item[c)] Na escala prática a salinidade é definida em termos da razão de
              condutividades elétricas da amostra de água do mar e de uma
              solução de cloreto de potássio.
    \item[d)] A salinidade na camada de superfície nos oceanos depende da
              evaporação, sendo por essa razão maior nas regiões equatoriais.
  \end{itemize}
\end{questions}

\begin{questions}
  Considere as Seções verticais sul-norte de temperatura potencial, salinidade,
  $\sigma_{\theta}$ e oxigênio dissolvido no Oceano Atlântico Oeste.

  \begin{parts}
    \part
    Tecer comentários sobre as referidas seções verticais.

    \part
    Esboçar perfis verticais de salinidade e densidade em baixas, médias e altas latitudes, comentando sobre suas características.
  \end{parts}
\end{questions}

\begin{questions}
  Considere a Equação Internacional de Estado (EOS-80), genericamente dada por:
  $\rho_{(S, T, p)} = \rho_{(S, T, 0)}(1 - p / K_{S,T, p})$

  \begin{parts}
    \part
    Discutir sobre o significado físico desta Equação e comentar, sucintamente,
    como a   mesma foi obtida.

    \part
    A partir da EIE-80 determinar a expressão para o cálculo do coeficiente de
    expansão térmica ($\alpha$).
  \end{parts}
\end{questions}

\begin{questions}
  Partindo da forma implícita da equação de estado da água do mar

  \[
    \rho  =  \rho(S, T, p)
  \]

  obter uma equação de estado na forma diferencial, em função dos coeficientes
  de expansão térmica ($\alpha$), contração salina ($\beta$) e compressibilidade
  bárica ($\kappa$).  Interpretar fisicamente a equação resultante.
\end{questions}

\begin{questions}
  \begin{parts}
    \part
    Baseado em seus conhecimentos, quais são as vantagens em se usar as formas
    simplificadas da equação de estado da água do mar?

    \part
    Como pesquisador, em quais situações ou regiões de estudo você usaria as
    formas simplificadas da equação de estado da água do mar.  Exemplificar.
  \end{parts}
\end{questions}
