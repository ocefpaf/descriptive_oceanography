% Style.
\documentclass[letterpaper,portuguese,12pt,pdftex]{exam}

\usepackage{setspace}
\usepackage{lineno}
% \usepackage[left=2.5cm,top=3cm,right=2.5cm]{geometry}
\usepackage[left=1.5cm,right=2.5cm,top=2cm,bottom=2cm]{geometry}

% Portuguese.
\usepackage[brazil]{babel}
\usepackage[T1]{fontenc}
\usepackage[utf8x]{inputenc}
\usepackage{textcomp}

% Font.
\usepackage{lmodern}

% Figures.
\usepackage{epsf,epsfig}

% Bibtex and extras.
\usepackage{natbib}
\usepackage{url}
\usepackage[bookmarks=false,colorlinks=true,urlcolor={green},linkcolor={green},pdfstartview={XYZ null null 1.22}]{hyperref}

% Math.
\usepackage{cancel}
\usepackage{mathtools}
\usepackage{amssymb,amsmath}
\everymath{\displaystyle}

\newcommand{\sen}{\operatorname{sen}}

% Exam.
\addpoints
\printanswers  % \noprintanswers
\usepackage{color}
\definecolor{SolutionColor}{rgb}{0.8,0.9,1}
\shadedsolutions
\renewcommand{\solutiontitle}{\noindent\textbf{Solução:}\par\noindent}
\pagestyle{headandfoot}
\footer{}{Página \thepage\ de \numpages}{}
\boxedpoints
\pointsinrightmargin
\pointpoints{ponto}{pontos}
\hqword{Questão}
\hpword{Pontos}
\hsword{Nota}
% \qformat{\textbf{Question\thequestion}\quad(\thepoints)\hfill}

% User commands.
\newcommand{\pd}[2]{\frac{\partial #1}{\partial #2}}

% PDF metadata.
\pdfinfo{% hyperref overrides this
  /Title    (Prova 03 -- Oceanografia Física Descritiva)
  /Author   (Filipe Fernandes)
  /Creator  (Filipe Fernandes)
  /Producer (Filipe Fernandes)
  /Subject  (prova)
  /Keywords (prova, oceanografia)
}

% Front page.
\title{Prova 03 -- Oceanografia Física Descritiva}
\author{Prof. Filipe Fernandes}
\date{20-Dez-2013}

\begin{document}
\maketitle
\doublespacing

\vspace{1cm}
\hbox to \textwidth{Nome e número de matrícula:\enspace\hrulefill}
\vspace{1cm}

\begin{minipage}{.8\textwidth}
Esse exame incluí \numquestions\ questões. O número total de pontos é \numpoints.
\vspace{1cm}

Essa prova segue o Acordo Ortográfico da Língua Portuguesa de 1990 (em vigor no
início de 2009).  Por isso erros de ortografia e gramática serão descontados da
sua nota final.

\vspace{1cm}

A prova deve ser feita individualmente e sem consulta.  O aluno deverá usar
CANETA (preta ou azul) para responder as questões – qualquer questão respondida
a lápis não será considerada na hora da correção.  Coloque seu nome em TODAS as
folhas e numere as mesma colocando o número total de folhas (Ex.: 1/4, 2/4 e
etc).

\vspace{1cm}

Leia atentamente todas as questões: a interpretação faz parte da prova e dúvidas
serão esclarecidas apenas após o término da mesma.

\end{minipage}

\clearpage

% 35
\begin{questions}
\question[5]
Discorra sobre a razão de {\bf aspecto dos oceanos} e como isso nos permite
{\bf simplificar} os movimentos que observamos.  Dê exemplo de pelo menos 1
fenômeno onde {\bf utilizamos} tal aproximação e comente pelo menos 1 fenômeno
onde ela se torna {\bf inválida}.

\begin{solution}
  A razão de aspecto dos oceanos é semelhante a uma folha de papel, onde a
  escala horizontal, de O(10$^{6}$ km), muito maior que que a escala vertical
  de O(1km).

  Essa aproximação é usada nos movimentos geostróficos onde a velocidade
  vertical é mínima quando comparada com as velocidades horizontais.  Porém não
  é válida em ressurgências costeiras, onde a velocidade vertical é de ordem
  semelhantes as correntes horizontais.
\end{solution}


\question[2\half]
Comente sobre a {\bf evolução} da medição de salinidade nos oceanos.  Sua
resposta deve englobar {\bf o que} o método media e com qual {\bf unidade} a
salinidade era representada.

\begin{solution}
  Para se medir salinidade absolutamente, em gramas de sal por quilogramas de
  água, apenas se pesarmos a água coletada, evaporar toda a água e medir os sais
  que sobram.  Devido a sua impraticabilidade Knudsen introduziu o método da
  Clorinidade, onde se mediam os cloretos de uma amostra e se calculava os sais
  totais em permil (\textperthousand{}).  Esse método por ser considerado
  ``quase'' direto porque mede apenas a componente principal dos sais e se
  calcula o resto por uma proporção conhecida e fixa para todos os oceanos.

  Com o advento do condutímetro passou-se a inferir salinidade indiretamente
  através da condutividade.  Esse método exige também uma coleta precisa dos
  dados de temperatura e pressão, pois a condutividade é uma função de S,T,P.
  As unidades de salinidade por condutividade evoluíram de:
  \begin{itemize}
    \item Unidade Prática de Salinidade (UPS) pela sua conversão de C,T,P para S.
    \item Sem unidades, porque oceanógrafos acreditavam que a inferência de S
          por C não contabilizava todos os sais e por isso não deviam ser
          expressa em unidades de ``sais.''
    \item Re-introdução da unidade de concentração de g kg$^{-1}$, por se usar
          um uma relação mais precisa entre C e os sais representados por ela
          devido a um maior conhecimento sobre a constituição da água do mar.
  \end{itemize}
\end{solution}


\question[2\half]
Sabendo que salinidade nos oceanos atualmente é representada em unidades de
g kg$^{-1}$ novamente e não mais em Unidades Práticas de Salinidade (UPS).
Calcule a salinidade para uma amostra com 630 g de água do mar, sendo e 22 g
de massa de sal.

\begin{solution}
  \begin{align*}
    S &= \dfrac{22 \text{g}}{630 \text{g}} \times \dfrac{10^{3} \text{g}}{1 \text{kg}} \\
    S &= 34.92 \text{ g kg}^{-1}
  \end{align*}
\end{solution}

\question[2]
O que é densidade e como podemos medi-lá? (Dica: pense com as unidades que
usamos para representá-la.)

\begin{solution}
  Densidade é a razão massa (kg) por volume (m$^3$).  Nos oceanos é muito
  complicado realizar medições direta de densidade ($\rho$).  Mas como $\rho$ é
  função de Salinidade, Temperatura e Pressão podemos estima-lá usando a equação
  de estado.
\end{solution}

\question[3]
Baseado no que se sabe do {\bf aquecimento desigual} da Terra.  Teça comentários
sobre a {\bf estratificação} nos oceanos e desenhe perfis de {\bf temperatura}
esquemáticos para Alta, Médias e Baixas latitudes.

\begin{solution}
  {\bf Resposta:} A Terra recebe as ondas curtas do Sol de forma desigual (mais
  concentrado no Equador com uma diluição progressiva em direção aos Polos).
  Isso resulta em um desbalanço de massa que é compensado com a circulação
  oceânica.  Essa circulação, em um modelo simples, levaria leva águas quentes
  em direção aos Polos na superfície e água frias dos Polos para o Equador em
  profundidade.

  {\bf Extra:} Porém, devido a distribuição de massa dos continentes, tamanho
  dos oceanos, variabilidades climáticas e etc, sabemos que há uma geração de
  massa ``fria'' maior no Hemisfério Norte do Atlântico (atualmente, em períodos
  passados esse cenário era diferente!), fazendo com que a água ``quente'' de
  superfície desse oceano ultrapasse o Equador e flua para Norte, contra
  intuitivo ao modelo simplista de balanço de calor do parágrafo anterior.
\end{solution}


\question[5]
A dinâmica de Ekman foi elaborada para explicar a deriva de icebergs no oceano
polar.  Porém, essa teoria se demostrou ser um resultado mais robusto que o
esperado e auxiliou oceanógrafos a explicarem diversos outros fenômenos
observados nos oceanos.

Cite {\bf 2} desses fenômenos e comente {\bf o papel} da dinâmica de Ekman em
cada um deles.

\begin{solution}
  A dinâmica de Ekman é uma teoria que balancei o atrito do vento (Tensão de
  Cisalhamento do Vento) com Coriolis (Força Fictícia em função da rotação da
  Terra).  O resultado é uma espiral em profundidade com velocidade decaindo e
  girando para direita (esquerda) no Hemisfério Norte (Sul).

  Essa solução, apesar de elegante, é muito teórica, com diversas aproximações
  irreais.  Adicionalmente ela não é prática nem ajuda a explicar os fenômenos
  observados.  Mas se ``somarmos'' as correntes da espiral obtemos o
  ``Transporte de Ekman'' integrado na coluna d'água.  Esse resultado é mais
  robusto  que a espiral e nos ajuda a explica de ressurgências costeiras até a
  circulação dos giros subtropicais.
\end{solution}

\question[5]
Diga se as alternativas abaixo são {\bf Verdadeiras} ou {\bf Falsas} e explique

Sobre o balanço geostrófico:

\begin{itemize}
  \item[a)] Não é um movimento forçado, por isso é uma força fictícia como
            Coriolis.
  \item[b)] É o balanço entre o atrito do vento e o a Força do Gradiente de
            Pressão.
  \item[c)] É um movimento estacionário.
  \item[d)] Se dá apenas quando Coriolis é máximo.
  \item[e)] Esse balanço só rege movimentos de pequena escala nos oceanos.
\end{itemize}


\begin{solution}
\begin{itemize}
  \item[a)] Falso. Não é um movimento forçado, mas não é força! Muito sim um
            balanço de Forças.
            Coriolis.
  \item[b)] Falso. É o balanço entre a Força do Gradiente de Pressão (FGP )e
            Força de Coriolis apenas.
  \item[c)] Verdadeira. É um movimento estacionário porque não é acelerado.  A
            força de Coriolis sempre se equipara a FGP.
  \item[d)] Falso.  A escala espaço temporal muda em função de Coriolis, mas
            ocorre até quando Coriolis é mínimo.
  \item[e)] Falso. Esse balanço só rege movimentos de Meso a Larga escala.
\end{itemize}
\end{solution}

\question[2\half]
Imagine que duas massas d'água com $\theta$=2 \textcelsius, S=35.04 g kg$^{-1}$ e
$\theta$=8.5\textcelsius, S=36.00 g kg$^{-1}$ se misturam em proporções
aproximadamente {\bf iguais}.  Qual será os valores de T e S da massa de água
que se origina dessa duas?

\begin{solution}
\begin{align*}
  \theta &= 5.25 ^\circ\text{C} \\
  S &= 35.52 \text{ g kg}^{-1}
\end{align*}
\end{solution}

\question[2\half]
  Explique os termos $a$, $b$ e $c$ da equação de estado em sua forma linear
  abaixo:

  \begin{equation}
    \rho = \rho_o - \underbrace{\alpha(T - T_o)}_{a} +
           \underbrace{\beta(S-S_o)}_b + \underbrace{\kappa p}_c
    \label{eq:linear}
  \end{equation}


  \begin{solution}
    $a$: Porção da densidade em função da temperatura. Expansão térmica
    ($\alpha$) e variação de temperatura ($\Delta T$).

    $b$: Porção da densidade em função da salinidade. Contração Salina
    ($\beta$) e variabilidades de salinidade ($\Delta S$).

    $c$: Porção da densidade em função da pressão. Compressão bárica ($\kappa$)
    e pressão ($p$).
  \end{solution}

\question[2\half]
Sobre a força/aceleração de Coriolis.  Indique se as alternativas abaixo são
{\bf verdadeira} ou {\bf falsas} e justifique.

\begin{itemize}
  \item[a)] A força de Coriolis é máxima na Equador.
  \item[b)] Como a força de Coriolis é fruto do produto vetorial entre a
            velocidade relativa da partícula e o sistema de coordenadas em
            rotação, apenas a componente vertical é importante para estudos do
            seu efeito na horizontal.
  \item[c)] A revolução da Terra é dada pelo dia sideral que leva em conta a
            rotação do sistema Terra, Sol e Via Láctea.
  \item[d)] O efeito da esfericidade da Terra é ignorado quando Coriolis é
            aproximado para um plano Cartesiano.
\end{itemize}

\begin{solution}
\begin{itemize}
  \item[a)] Falsa. A força de Coriolis é {\bf mínima} na Equador.
  \item[b)] Verdadeira, por isso retemos apenas $fu$ e $fv$ nas eqs.
  \item[c)] Falsa. A revolução da Terra é dada pelo dia sideral que leva em
            conta a rotação do sistema Terra-Sol.
  \item[d)] Falsa. O efeito da esfericidade da Terra é aproximado para $\beta y$
            em um plano Cartesiano ($f = f_o + \beta y$).
\end{itemize}
\end{solution}

\question[2\half]
Sobre Correntes nos oceanos.  Indique se as alternativas abaixo são
{\bf verdadeira} ou {\bf falsas} e justifique.

\begin{itemize}
  \item[a)] Correntes são definidas pela medida instantânea das velocidades das
            nos oceanos.
  \item[b)] As Correntes geradas pelo vento colocam todos os oceanos em
            conexão como uma grande ``esteira de aeroporto'' global.
  \item[c)] As correntes costeiras e correntes oceânicas têm escalas
            temporais mais curtas e longas respectivamente.
  \item[d)] Correntes termohalinas fecham o balanço de Sal e Calor do globo
            equilibrando a distribuição dos mesmo.
\end{itemize}

\begin{solution}
\begin{itemize}
  \item[a)] Falso. Correntes são definidas pela medida em certa escala temporal
            definida.
  \item[b)] Falso. As Correntes termohalinas colocam todos os oceanos em
            conexão como uma grande ``esteira de aeroporto'' global.  As
            correntes geradas pelo ventos fecham os giros das bacias oceânicas
            e $\approx$ 1000 metros de profundidade.
  \item[c)] Verdadeiro. As correntes costeiras estão restritas as plataforma
            continentais que por sua vez é << que as bacias oceânicas.
  \item[d)] Verdadeira. Correntes termohalinas fecham o balanço de Sal e Calor do globo
            equilibrando a distribuição dos mesmo.
\end{itemize}
\end{solution}

\end{questions}
\end{document}
