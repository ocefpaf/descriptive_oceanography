% Style.
\documentclass[letterpaper,portuguese,12pt,pdftex]{exam}

\usepackage{setspace}
\usepackage{lineno}
\usepackage[left=2.5cm,top=3cm,right=2.5cm]{geometry}

% Portuguese.
\usepackage[brazil]{babel}
\usepackage[T1]{fontenc}
\usepackage[utf8x]{inputenc}
\usepackage{textcomp}

% Font.
\usepackage{lmodern}

% Figures.
\usepackage{epsf,epsfig}

% Bibtex and extras.
\usepackage{natbib}
\usepackage{url}
\usepackage[bookmarks=false,colorlinks=true,urlcolor={green},linkcolor={green},pdfstartview={XYZ null null 1.22}]{hyperref}

% Math.
\usepackage{cancel}
\usepackage{mathtools}
\usepackage{amssymb,amsmath}
\everymath{\displaystyle}

\newcommand{\sen}{\operatorname{sen}}

% Exam.
\addpoints
\printanswers  % \noprintanswers
\usepackage{color}
\definecolor{SolutionColor}{rgb}{0.8,0.9,1}
\shadedsolutions
\renewcommand{\solutiontitle}{\noindent\textbf{Solução:}\par\noindent}
\pagestyle{headandfoot}
\footer{}{Página \thepage\ de \numpages}{}
\boxedpoints
\pointsinrightmargin
\pointpoints{ponto}{pontos}
\hqword{Questão}
\hpword{Pontos}
\hsword{Nota}
% \qformat{\textbf{Question\thequestion}\quad(\thepoints)\hfill}

% User commands.
\newcommand{\pd}[2]{\frac{\partial #1}{\partial #2}}

% PDF metadata.
\pdfinfo{% hyperref overrides this
  /Title    (Prova 02 -- Oceanografia Física Descritiva)
  /Author   (Filipe Fernandes)
  /Creator  (Filipe Fernandes)
  /Producer (Filipe Fernandes)
  /Subject  (prova)
  /Keywords (prova, oceanografia)
}

% Front page.
\title{Prova 02 -- Oceanografia Física Descritiva}
\author{Prof. Filipe Fernandes}
\date{06-Dez-2013}

\begin{document}
\maketitle
\doublespacing

\vspace{1cm}
\hbox to \textwidth{Nome e número de matrícula:\enspace\hrulefill}
\vspace{1cm}

\begin{minipage}{.8\textwidth}
Esse exame incluí \numquestions\ questões. O número total de pontos é \numpoints.
\vspace{1cm}

Essa prova segue o Acordo Ortográfico da Língua Portuguesa de 1990 (em vigor no
início de 2009).  Por isso erros de ortografia e gramática serão descontados da
sua nota final.

A prova deve ser feita individualmente e sem consulta.  O aluno deverá usar
CANETA (preta ou azul) para responder as questões – qualquer questão respondida
à lápis não será considerada na hora da correção.  Coloque seu nome em TODAS as
folhas e numere as mesma colocando o número total de folhas (Ex.: 1/4, 2/4 e
etc).


\vspace{1cm}

Leia atentamente todas as questões: a interpretação faz parte da prova e dúvidas
serão esclarecidas apenas após o término da mesma.


\end{minipage}

\clearpage


\begin{questions}

\question[5]
Explique o que são {\bf massas d'água}.  Como são {\bf formadas} e como podemos
{\bf rastreá-las} nos oceanos.  Faça uso da equação abaixo em sua explicação.
Qual termo {\bf não} pode existir para que a propriedade em questão seja
{\bf conservativa}.

\[
  \underbrace{\pd{S}{t}}_a = \underbrace{\vec{u}\nabla S}_b + \underbrace{\phi}_c
\]

\begin{solution}
 Massas d'água é uma porção de água do mar com uma origem determinada.
 Formam-se na superfície do mar devido às forças atmosféricas ganhando o seu $T$
 e $S$ característico pelo qual podemos rastreá-las.  Na equação temos que
 cortar o termo que corresponde à fonte ou sorvedouro $\phi$.
\end{solution}


\question[3]
Sobre a força/aceleração de Coriolis.  Indique se as alternativas abaixo são
VERDADEIRA ou FALSAS e justifique.
\begin{itemize}
  \item[a)] A força de Coriolis é máxima na Equador.
  \item[b)] Como a força de Coriolis é fruto do produto VETORIAL entre a velocidade
        relativa da partícula e o sistema de coordenadas em rotação, apenas a
        componente vertical é importante para estudos do seu efeito na
        horizontal.
  \item[c)] A revolução da Terra é dada pelo dia sideral que leva em conta a rotação
        do sistema Terra, Sol e Via Láctea.
  \item[d)] O efeito da esfericidade da Terra é ignorado quando Coriolis é
  aproximado para um plano Cartesiano.
\end{itemize}

\begin{solution}
\begin{itemize}
  \item[a)] Falsa. A força de Coriolis é {\bf mínima} na Equador.
  \item[b)] Verdadeira, por isso retemos apenas fu e fv nas eqs.
  \item[c)] Falsa. A revolução da Terra é dada pelo dia sideral que leva em
            conta a rotação do sistema Terra-Sol.
  \item[d)] Falsa. O efeito da esfericidade da Terra é aproximado para $\beta y$
            um plano Cartesiano ($f = f_o + \beta y$).
\end{itemize}
\end{solution}



\question
A figura \ref{fig:giro} é uma representação esquemática de um típico {\bf Giro
Subtropical}, como os que formam as correntes do Brasil e corrente do Golfo.


\begin{figure}[ht]
  \centering
  \includegraphics{./figures/agua_central.png}
  \caption{Giro Subtropical.}
  \label{fig:giro}
\end{figure}

\begin{parts}
  \part[3]
  Explique como se dá o ``{\bf transporte de Ekman}'' assinalado pelas setas
  transparentes.  Sua resposta deve conter o {\bf balanço de forças},
  {\bf espiral de Ekman} e {\bf transporte} integrado na coluna d'água.

  \begin{solution}
   O transporte de Ekman se da pelo balanço entre as forças de atrito e
   Coriolis, é  o transporte integrado na coluna d'água.  Ou seja, é o somatório
   da espiral na profundidade da influência de Ekman.

   Na figura o transporte está sempre a direita dos ventos (por isso deve ser
   no Hemisfério Norte), gerando uma convergência de água no centro do giro.
  \end{solution}

  \part[1]
  Usando conceitos de {\bf convergência} e {\bf divergência}, explique como o
  transporte de  Ekman gera a {\bf alta pressão} do Giro Subtropical.  Qual é o
  papel desse fenômeno na formação da {\bf água central}?

  \begin{solution}
   Essa convergência gera uma alta pressão que acarreta em uma subsidência
   (apontada pela seta de bombeamento de Ekman).  Essa subsidência é essencial
   no processo de formação da Água Central, pois essa é formada o ano todo
   devido a ``ajuda'' mecânica de Ekman.  Por outro lado, as massas de fundo e
   intermediárias só se foram durante o inverno quando há convecção.
  \end{solution}

\end{parts}

\question[3]
A corrente do Brasil nas proximidades de 20\textdegree{S} (latitude) tem
velocidades superficiais na ordem de -0,65 m s$^{-1}$ e uma largura média de
$\sim$ 100 km.

Estime a altura da superfície livre ($\Delta h$) da corrente do Brasil usando a
relação {\bf geostrófica}:
 \[
    fv = g\tan(\theta),
 \]


onde:
\begin{itemize}
 \item $v$ é a velocidade meridional;
 \item $f$ o parâmetro de Coriolis dado por $2\Omega\sen(\text{lat})$ e,
  \begin{enumerate}
   \item[] $\Omega = 7.29 \times 10^{-5}$ rad s$^{-1}$;
   \item[] lat $\equiv$ latitude local;
  \end{enumerate}

 \item $g$ a aceleração da gravidade;
 \item $\theta$ o ângulo formado pela declinação da superfície livre.
\end{itemize}

\begin{figure}[ht]
  \centering
  \includegraphics[scale=0.7]{./figures/BC_slope.png}
  \caption{Desenho esquemático da declinação gerada pela alta pressão do giro
           Subtropical.}
  \label{fig:BC}
\end{figure}


\begin{solution}
  Exato:
\begin{align*}
  f &= 2\Omega\sen\left( \dfrac{-20 \pi}{180} \right) \\
  f &= 0.0001458423 \sen(-0.349)\\
  f &= -4.988\times 10^{-5} \text { s}^{-1}\\
  &\downarrow\\
 \tan\phi &= \frac{fv}{g}\\
 \tan\phi &= \frac{-4.988\times 10^{-5} \text { s}^{-1} \times -0.65 \text{ m s}^{-1}}{9.8 \text{ m s}^{-2}}\\
  \phi &= \arctan({3.308 \times 10^{-6}})\\
  \phi &=0.000189^{\circ}\\
  \text{ ou,}\\
 \Delta h &= 10^5\text{ m} \times 3.308 \times 10^{-6}\\
 \Delta h &= 0.33 \text{ m}
\end{align*}

  Ordem de grandeza:
\begin{align*}
  f &= -10^{-5} \text { s}^{-1} \\
  g &= 10 \text{ m s}^{-2} \\
  v & = -10^{-1} \text{ m s}^{-1} \\
  &\downarrow \\
 \tan\phi &= \frac{fv}{g} \\
 \tan\phi &= \frac{-10^{-5} \cancel{\text { s}^{-1}} \times -10^{-1} \cancel{\text{ m s}^{-1}}}{10 \cancel{\text{ m s}^{-2}}} \\
 \tan\phi &= \frac{-10^{-6}}{10} = -10^{-7} \\
 \frac{\Delta h}{\Delta x} &= 10^{-7} \\
 \Delta h &= 10^5\text{ m} \times 10^{-7} \\
  \Delta h &= 10^{-2}\text{ m} \\
\end{align*}
Ou seja na ordem de centímetros.

\end{solution}

\question
Extra (5 pontos).
Discorra sobre a definição do que são correntes oceânicas, e fale sobre as
diferenças da circulação termohalina e gerada pelo vento.

\end{questions}
\end{document}
